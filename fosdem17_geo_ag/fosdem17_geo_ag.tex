% Build with xetex.
\documentclass[aspectratio=169,handout,bigger]{beamer}

%% -------------------------------------------------------------------------- %%

\usepackage{xunicode}
\usepackage{xltxtra}
\usepackage{graphicx}

\usepackage{color}

\usepackage{setspace}
\usepackage{ragged2e}

\usepackage[normalem]{ulem}

\newcommand{\soutt}[1]{%
    \renewcommand{\ULthickness}{1pt}%
       \sout{#1}%
    \renewcommand{\ULthickness}{.4pt}% Resetting to ulem default
}

\usepackage{minted}

\usepackage{tikz}
\usetikzlibrary{shapes,arrows,chains,calc,fit,matrix}

%% -------------------------------------------------------------------------- %%

\usepackage{polyglossia}
\setmainlanguage{russian}
\setotherlanguage{english}

% NB: To get MS fonts for OS X:
%
% 1) Install MS Open XML Converter.
% 2) Install OpenXML_all_fonts.pkg from inside of Converter's mpkg.
%
% http://www.labnol.org/software/tutorials/\
% free-download-calibri-font-on-mac/3684/

\setmainfont{Calibri}
\setsansfont{Calibri}
\setmonofont{Consolas}

%% ========================================================================== %%

\useinnertheme{circles}

\setbeamertemplate{navigation symbols}{}
\setbeamertemplate{section in toc}[sections numbered]

\definecolor{chart11}{RGB}{0, 0, 0}
\setbeamercolor{title}{fg=chart11}
\setbeamercolor{author}{fg=chart11}
\setbeamercolor{frametitle}{fg=chart11}
\setbeamercolor{itemize item}{fg=chart11}
\setbeamercolor{itemize subitem}{fg=chart11}
\setbeamercolor{itemize subsubitem}{fg=chart11}
\setbeamercolor{section in toc}{fg=chart11}

\setbeamerfont{title}{series=\bfseries,parent=structure}

%% ========================================================================== %%

\title{Rapid backend prototyping for a geolocation-based mobile game}
\subtitle{With OpenResty, Redis and Docker}
\author{\includegraphics[height=.4\textheight]{logo}}
\institute{Alexander Gladysh <ag@logiceditor.com>\newline@agladysh}
\date{FOSDEM 2017}

%% ========================================================================== %%

\begin{document}

% Using [plain] to avoid frame number on the title page.
\begin{frame}[plain]
 \titlepage
\end{frame}

%% ========================================================================== %%

\begin{frame}{Talk plan}

\tableofcontents

\end{frame}

%% ========================================================================== %%

\section*{}

\begin{frame}{About me}

\begin{itemize}
\item Developing software since 2002
\item Most of the time in gamedev
\item Beyond that: high-load internet solutions, enterprise software etc. etc.
\item Organizer of
      \href{http://meetup.com/Lua-in-Moscow}{meetup.com/Lua-in-Moscow}. \\
      Come to our Lua-related conference in Moscow on March 5!
\end{itemize}

\end{frame}

%% ========================================================================== %%

\section{The Case}

%% ========================================================================== %%

\begin{frame}{Mobile games with geolocation}
  \centering{\includegraphics[height=.75\textheight]{pokemongo}}
\end{frame}

%% -------------------------------------------------------------------------- %%

\begin{frame}{The Goals}
  \begin{itemize}
    \item To try out a number of approaches to the gameplay,
          to generate new ideas,
          to figure out what is fun and what is not.
    \item To get as cheaply as possible the framework that would allow
          to iterate over gameplay variants as fast as possible.
    \item To figure out technical limitations of the genre in practice.
  \end{itemize}
\end{frame}

%% -------------------------------------------------------------------------- %%

\begin{frame}{Results}
  A geolocation game server-side prototype along with a rudimentary
  client-side was developed in less than 100 man-hours (2 calendar months)
  \vspace*{1em}\par
  We're rapidly iterating over the gameplay options
  and develop the technology part of the project.
\end{frame}

%% -------------------------------------------------------------------------- %%

\begin{frame}{What is this talk about?}
  \begin{itemize}
    \item This talk is about the technology part of the project,
    \item not about game-design
    \item or monetization.
  \end{itemize}
  \vspace*{1em}\par
  It now is easier when ever to develop geolocation-based games.
  \vspace*{1em}\par
  I will show where to begin.
\end{frame}

%% ========================================================================== %%

\section{The Stack}

%% ========================================================================== %%

\begin{frame}{First prototype gameplay}
  \begin{itemize}
    \item The player is searching for
          the mobs placed on a map by walking around;
    \item Player has a set chance to catch the found mob.
    \item Caught mobs increase a stats counter in player characteristics.
    \item Caught mobs disappear from the map,
          but respawn at the same place after a set time.
    \item Admin users may add new mobs on the map.
  \end{itemize}
\end{frame}

%% -------------------------------------------------------------------------- %%

\begin{frame}{The Stack}
  \begin{itemize}
    \item Server:
    \begin{itemize}
      \item Redis,
      \item OpenResty,
      \item Docker.
    \end{itemize}
    \item Client:
    \begin{itemize}
      \item A single-page web application (in browser),
      \item HTML5.
    \end{itemize}
  \end{itemize}
\end{frame}

%% -------------------------------------------------------------------------- %%

\begin{frame}{Not Invented Here Syndrome?}
  Yes and No.
\end{frame}

%% -------------------------------------------------------------------------- %%

\begin{frame}{Redis}
  \begin{itemize}
    \item A reliable, proven solution.
    \item Supports geoposition out of the box:
    \begin{itemize}
      \item \texttt{GEOADD key longitude latitude member}
      \item \texttt{GEORADIUS key longitude latitude radius m}
    \end{itemize}
    \item Useful set of primitives to store game objects in.
    \item "Stored procedures" in Lua.
  \end{itemize}
\end{frame}

%% -------------------------------------------------------------------------- %%

\begin{frame}{OpenResty}
  \begin{itemize}
    \item An nginx distributive supporting
          Lua, Redis and many other things out of the box.
    \item Very fast, rather friendly and well-maintained.
    \item Useful both for quick prototyping and production environment.
  \end{itemize}
\end{frame}

%% -------------------------------------------------------------------------- %%

\begin{frame}{Docker}
  \begin{itemize}
    \item Reproducible cross-platform development environment.
    \item Efficiently relieves the development environment pain set up.
    \item Can be quickly turned into a prototype of production environment
          (it is arguable if it is suitable for real production).
    \item Has to be updated to a sufficiently recent version.
  \end{itemize}
\end{frame}

%% -------------------------------------------------------------------------- %%

\begin{frame}{Browser, HTML5}
  \begin{itemize}
    \item Server-side during early stages of development is more important.
    \item Don't implement what can be imagined while play-testing,
          like augmented-reality combat and other frills.
    \item HTML5 is good to implement "quick and dirty" client.
    \item NB: Geolocation data access is limited, but sufficient.
  \end{itemize}
\end{frame}

%% ========================================================================== %%

\section{Docker}

%% ========================================================================== %%

\begin{frame}{Docker: how to install on Ubuntu}
  \begin{itemize}
    \item Traditionally, Ubuntu version is outdated.
    \item Don't do \texttt{wget | sh}.
    \item Install \texttt{docker} и \texttt{docker-machine}
          from docker's apt-repository (see manual).
    \item \texttt{docker-compose} is to be installed by \texttt{pip install}.
    \item Read the manual for the installation on other platforms.
  \end{itemize}
\end{frame}

%% -------------------------------------------------------------------------- %%

\begin{frame}{Docker on the developer's machine}

\begin{center}
\begin{tikzpicture}
  every join/.style={line},    % Default linetype for connecting boxes
  scale=1, transform shape
]

\tikzset{
  basebase/.style={
    align=center, minimum height=2em, minimum width=9em, color=chart11,
    inner sep=.5em, node distance=3em and 11em
  },
  rect/.style={
    basebase, draw, on grid
  },
  coord/.style={coordinate},
  line/.style={->, draw, chart11},
}

\node[rect, rounded corners]                     (client)    {Client};
\node[rect, rounded corners, below of=client]    (docker)    {localhost:8080};
\node[rect, below of=docker]    (openresty) {OpenResty};
\node[rect, below of=openresty] (redis)     {Redis};

\draw [line] (client)    -- (docker);
\draw [line] (docker)    -- (openresty);
\draw [line] (openresty) -- (redis);

\end{tikzpicture}
\end{center}

\end{frame}

%% -------------------------------------------------------------------------- %%

\begin{frame}[fragile]{docker-compose.yml for development: Redis}
\begin{minted}{text}
version: "2"
services:
  redis:
    image: redis
    volumes:
      - ./redis:/data
    command: redis-server --appendonly yes
  openresty: <...>
\end{minted}
\end{frame}

%% -------------------------------------------------------------------------- %%

\begin{frame}[fragile]{OpenResty: interesting parts of nginx.conf (reduced)}
\begin{minted}{nginx}
error_log logs/error.log notice;
http {
  include resolvers.conf;
  lua_package_path "$prefix/lualib/?.lua;;";
  lua_code_cache off; # TODO: Enable on production!
  server {
    listen 8080;
    include mime.types;
    default_type application/json;
    location / { index index.html; root static/; }
    location = /api/v1/ { content_by_lua_file 'api/index.lua'; }
  }
}
\end{minted}
\end{frame}

%% -------------------------------------------------------------------------- %%

\begin{frame}[fragile]{OpenResty: Dockerfile}
\begin{minted}{docker}
FROM openresty/openresty
COPY bin/entrypoint.sh /usr/local/bin/openresty-entrypoint.sh
COPY nginx/conf /usr/local/openresty/nginx/conf
COPY nginx/lualib /usr/loca/openresty/nginx/lualib
COPY nginx/lua /usr/loca/openresty/nginx/lua
COPY nginx/static /usr/loca/openresty/nginx/static
ENTRYPOINT /usr/local/bin/openresty-entrypoint.sh
\end{minted}
\end{frame}

%% -------------------------------------------------------------------------- %%

\begin{frame}[fragile]{OpenResty: entrypoint.sh}
\begin{minted}{bash}
#!/bin/sh
grep nameserver /etc/resolv.conf \
  | awk '{print  "resolver " $2 ";"}' \
  > /usr/local/openresty/nginx/conf/resolvers.conf
/usr/local/openresty/bin/openresty -g 'daemon off;' "$@"
\end{minted}
\end{frame}

%% -------------------------------------------------------------------------- %%

\begin{frame}[fragile]{docker-compose.yml for development: OpenResty}
\begin{minted}{text}
  <...>
  openresty:
    build: .
    ports:
      - "8080:8080"
    volumes:
      - ./nginx/lualib:/usr/local/openresty/nginx/lualib:ro
      - ./nginx/api:/usr/local/openresty/nginx/api:ro
      - ./nginx/static:/usr/local/openresty/nginx/static:ro
    links:
      - redis
\end{minted}
\end{frame}

%% ========================================================================== %%

\section{HTTP API}

%% ========================================================================== %%

\begin{frame}{API calls}
  \begin{itemize}
    \item Пользовательские вызовы:
      \begin{itemize}
        \item \texttt{/} состояние игрового мира,
        \item \texttt{/go/:go-id/} состояние игрового объекта,
        \item \texttt{/go/:go-id/act/:action-id} выполнение действия.
      \end{itemize}
    \item Системные вызовы:
    \begin{itemize}
      \item \texttt{/register} создание пользователя,
      \item \texttt{/reset} сброс базы в исходное состояние,
      \item \texttt{/patch} апгрейд базы до текущей версии.
    \end{itemize}
    \item NB: Админку (бэкофис) не делаем,
          используем внутриигровые механики для администрирования игрового мира.
  \end{itemize}
\end{frame}

%% ========================================================================== %%

\section{Устройство игрового мира}

%% ========================================================================== %%

\begin{frame}{Игровой объект}
  \begin{itemize}
    \item С точки зрения сервера игровой мир состоит из игровых объектов.
    \item Игровой объект имеет численные характеристики и действия.
    \item Игровые объекты могут иметь координаты.
    \item Игровые объекты без координат должны принадлежать другим объектам
          или быть их прототипами.
  \end{itemize}
\end{frame}

%% -------------------------------------------------------------------------- %%

\begin{frame}{Цепочка прототипов}
  \begin{itemize}
    \item Игровой объект может иметь прототип.
    \item Игровой объект --- прототип в свою очередь также может иметь прототип.
    \item Игровой объект наследует характеристики и действия своих прототипов.
  \end{itemize}
\end{frame}

%% -------------------------------------------------------------------------- %%

\begin{frame}{Характеристики}
  \begin{itemize}
    \item Характеристика --- именованное численное свойство игрового объекта.
    \item Если у игрового объекта нет какой-то характеристики,
          её значение берётся у ближайшего прототипа по цепочке
          (если не нашли --- 0).
  \end{itemize}
\end{frame}

%% -------------------------------------------------------------------------- %%

\begin{frame}{Действия}
  \begin{itemize}
    \item Действие на игровом объекте --- идентификатор из таблицы
          обработчиков действий.
    \item Действие может быть инициировано игроком,
          если у него достаточно на это прав.
  \end{itemize}
\end{frame}

%% -------------------------------------------------------------------------- %%

\begin{frame}[fragile]{Моб: Зелёная Жаба}
\begin{minted}{lua}
{
  id = 'proto.mob.collectable';
  chrs = { respawn_dt = 10 * 60 };
  actions = { 'mob.collect' };
};
{
  id = 'proto.mob.toad.green';
  proto_id = 'proto.mob.collectable';
  chrs = { escape_chance = 0.25 };
};
{
  id = 'fa2eb7bca46c11e6be447831c1cebc82';
  proto_id = 'proto.mob.toad.green';
  geo = { lat = 55.7558, lon = 37.6173 };
};
\end{minted}
\end{frame}

%% -------------------------------------------------------------------------- %%

\begin{frame}[fragile]{Действие: поймать моба}
\begin{minted}{lua}
ACTIONS['mob.collect'] = function(target, initiator)
  if
    math.random() * initiator.chrs.collect_skill >
    target.chrs.escape_chance
  then
    -- Inc number of catches for this mob type
    go_inc_chr(initiator.id, target.proto_id, 1)
    go_schedule_action_initiation( -- Schedule respawn
      target.chrs.respawn_dt, 'mob.spawn',
      { proto_id = target.proto_id, pos = target.pos },
      initiator.id
    )
    go_remove(target.id) -- Mob is caught, remove
  end
end
\end{minted}
\end{frame}

%% -------------------------------------------------------------------------- %%

\begin{frame}[fragile]{Выполнение отложенных действий}
\begin{minted}{lua}
local timestamp = os.time()
local action_ids = redis:zrangebyscore('da', '-inf', timestamp)
for i = 1, #action_ids do
  -- Execute action_ids[i] action
end
redis:zremrangebyscore('da', '-inf', timestamp)
\end{minted}
\end{frame}

%% -------------------------------------------------------------------------- %%

\begin{frame}[fragile]{Игрок}
\begin{minted}{lua}
{
  id = 'proto.user';
  chrs = { vision = 100, reach = 50 };
};
{
  id = 'user.1';
  geo = { lat = 55.7558, lon = 37.6173 };
  chrs = { collect_skill = 0.5 };
};
\end{minted}
\end{frame}

%% -------------------------------------------------------------------------- %%

\begin{frame}[fragile]{Предмет: Админская шапка}
\begin{minted}{lua}
{
  id = 'proto.item.wearable';
  actions = {
    ['item.don'] = { enabled = true };
    ['item.doff'] = { enabled = false };
  };
}
{
  id = 'proto.item.admin-hat';
  proto_id = 'proto.item.wearable';
  grants = { 'user.admin' };
  chrs = { collect_skill = 0.25 };
};
\end{minted}
\end{frame}

%% -------------------------------------------------------------------------- %%

\begin{frame}[fragile]{Выдадим админскую шапку пользователю}
\begin{minted}{lua}
local hat = go_new('proto.item.admin-hat')

assert(go_get('user.1').stored[1] == nil)

go_store('user.1', hat.id)

assert(go_get('user.1').stored[1] == hat.id)
\end{minted}
\end{frame}

%% -------------------------------------------------------------------------- %%

\begin{frame}{Хранение ("storage")}
  \begin{itemize}
    \item Игровой объект может "хранить" другие объекты.
    \item Хранимые объекты не "видны" извне хранящего объекта.
    \item Пользователю доступны действия
          непосредственно хранимых им объектов.
    \item Характеристики хранимых объектов никак не влияют
          на характеристики хранящих их объектов.
  \end{itemize}
\end{frame}

%% -------------------------------------------------------------------------- %%

\begin{frame}[fragile]{Действия на админской шапке}
\begin{minted}{lua}
ACTIONS['item.don'] = function(target, initiator)
  go_unstore(initiator.id, target.id)
  go_attach(initiator.id, target.id)
  go_disable_action(target.id, 'item.don')
  go_enable_action(target.id, 'item.doff')
end

ACTIONS['item.doff'] = function(target, initiator)
  go_attach(initiator.id, target.id)
  go_store(initiator.id, target.id)
  go_disable_action(target.id, 'item.doff')
  go_enable_action(target.id, 'item.don')
end
\end{minted}
\end{frame}

%% -------------------------------------------------------------------------- %%

\begin{frame}{Прикрепление / надевание ("attachment")}
  \begin{itemize}
    \item К игровому объекту могут быть "прикреплены" другие объекты.
    \item Прикреплённые объекты видны извне родительского объекта.
    \item Пользователю доступны действия
          прикреплённых непосредственно к нему объектов.
    \item Характеристики прикреплённых объектов прибавляются
          к характеристикам родительских объектов.
  \end{itemize}
\end{frame}

%% -------------------------------------------------------------------------- %%

\begin{frame}[fragile]{Предмет: Спавнилка зелёных жаб}
\begin{minted}{lua}
{
  id = 'proto.item.spawner.toad.green';
  actions = {
    ['mob.spawn'] = {
      requires = { 'user.admin' };
      param = { proto_id = 'proto.mob.toad.green' };
    };
  };
};
\end{minted}
\end{frame}

%% -------------------------------------------------------------------------- %%

\begin{frame}{Права на действия}
  \begin{itemize}
    \item Действие доступно для выполнения только если \texttt{grants} игрока
          содержит все записи из \texttt{requires} действия.
    \item Прикреплённые к игроку ("надетые") предметы
          добавляют ему свои \texttt{grants}.
  \end{itemize}
\end{frame}

%% ========================================================================== %%

\section{Demo}

%% ========================================================================== %%

\begin{frame}{Demo}
  \begin{itemize}
    \item Current version of the application can be found here
          \href{https://geo.logiceditor.com/}{geo.logiceditor.com}.
    \item Sources are linked to at the same page.
    \item The very first client is always \texttt{curl}.
          You can try out the API using it by appending
          \texttt{/api/v1} to the URL of the application.
  \end{itemize}
\end{frame}

%% ========================================================================== %%

\section{Клиент}

%% ========================================================================== %%

\begin{frame}{Как работает клиент?}
  \begin{itemize}
    \item Инициализируются геолокация и гуглекарты.
    \item Текущая позиция отправляется на сервер.
    \item Сервер возвращает перечень видимых объектов с возможными действиями.
    \item Объекты помечаются маркерами на карте и выводятся под ней вёрсткой.
    \item Ожидаем активации действия пользователем либо смены координат.
  \end{itemize}
  \vspace*{1em}\par
  NB:
  \begin{itemize}
    \item Для генерации имён жаб на основе их идентификаторов
          используется chance.js.
    \item Перерисовку лучше проводить по таймеру,
          вне зависимости от цикла обновления данных.
  \end{itemize}
\end{frame}

%% -------------------------------------------------------------------------- %%

\begin{frame}{Геолокация на HTML5}
  \begin{itemize}
    \item \texttt{navigator.geolocation.watchPosition(callback, options)}.
    \item В Chrome пользуйтесь панелью разработчика Sensors для
          отладки геолокации.
    \item В Chrome по соображением безопасности отключена геолокация
          для протокола HTTP (за исключением сайтов на localhost).
          Используйте HTTPS, например, с сертификатами от Let's Encrypt.
  \end{itemize}
\end{frame}

%% -------------------------------------------------------------------------- %%

\begin{frame}[fragile]{Google Maps}
\begin{minted}{js}
new google.maps.Map(assert(document.getElementById('map')), {
  center: new google.maps.LatLng(pos.lat, pos.lon),
  zoom: 18,
  mapTypeId: google.maps.MapTypeId.ROADMAP,
  disableDefaultUI: true,
  disableDoubleClickZoom: true,
  draggable: false,
  scrollwheel: false,
  styles: [ { featureType: "poi",
    stylers: [ { visibility: "off" } ]
  } ]
});
\end{minted}
\end{frame}

%% ========================================================================== %%

\section{Итоги}

%% ========================================================================== %%

\begin{frame}{Проблемы проекта}
  \begin{itemize}
    \item Шумные данные от GPS.
    \item Крайне низкая точность геолокации в зданиях.
    \item ...
  \end{itemize}
\end{frame}

%% -------------------------------------------------------------------------- %%

\begin{frame}{\soutt{Проблемы} Технические ограничения проекта}
  \begin{itemize}
    \item Шумные данные от GPS.
    \item Крайне низкая точность геолокации в зданиях.
    \item ...
  \end{itemize}

  Нет проблем. Есть ограничения, под которые нужно подстраивать геймплей.
  Часть из них решается технологически. Нужно ли тратить время на это решение
  --- один из вопросов, на которые должен ответить этап препродакшена.
\end{frame}

%% -------------------------------------------------------------------------- %%

\begin{frame}{Недостающие механизмы}
  \begin{itemize}
    \item Самое крупное --- система событий.
    \item Много мелких функций, например счётчик пройденных метров.
  \end{itemize}
\end{frame}

%% -------------------------------------------------------------------------- %%

\begin{frame}{Ошибки}
  \begin{itemize}
    \item Поздновато добавили геолокациию.
    \item Поздновато добавили карту в клиенте.
    \item Мало выходили на улицу чтобы тестировать.
    \item ...
    \item Найдите сами, сравнив код на слайдах с кодом в проекте.
  \end{itemize}
\end{frame}

%% -------------------------------------------------------------------------- %%

\begin{frame}{Итоги}
  \begin{itemize}
    \item Относительно малыми усилиями
    \item мы сделали крошечную мобильную игру с геолокацией
    \item и заложили фундамент для быстрой разработки большого числа
          несложных прототипов
    \item для поиска удачных вариантов геймплея в этом жанре.
  \end{itemize}
\end{frame}

%% -------------------------------------------------------------------------- %%

\begin{frame}{Благодаря чему разработка быстрая?}
  \begin{itemize}
    \item В проекте небольшой объём простого кода,
    \item использующего базовые механизмы и надёжные сторонние решения
    \item для решения большого числа поставленных геймдизайнером задач.
  \end{itemize}

  \begin{itemize}
    \item Аккуратное расширение возможностей этого кода
    \item ещё больше расширит круг задач, которые можно будет решить,
    \item поменяв несколько строк в конфиге.
  \end{itemize}
\end{frame}

%% -------------------------------------------------------------------------- %%

\begin{frame}{Векторы развития проекта}
  \begin{itemize}
    \item \Huge{Новые варианты геймплея}
    \item \large{Новые функции и механизмы}
    \item \normalsize{Решение технических проблем}
  \end{itemize}
\end{frame}

%% -------------------------------------------------------------------------- %%

\begin{frame}{Как работать с гейм-дизайнером на ранних этапах прототипирования?}
  \begin{itemize}
    \item Быстро итерироваться.
          В идеале --- садиться рядом, кодить и сразу получать фидбек.
          Пробовать самому иногда делать рутинную часть работы гейм-дизайнера,
          чтобы лучше понять, что именно мешает и тормозит процесс.
    \item В первую очередь исправлять мешающие тестировать геймплей баги,
          потом улучшать старые механизмы в коде и добавлять новые.
    \item Если для новой геймплейной фичи нет механизма, добавлять его,
          хотя бы в самой грубой форме, а не реализовывать решение в лоб.
    \item Все прочие задуманные изменения в коде, в том числе рефакторинг,
          реализовывать в порядке убывания боли от их отсутствия.
  \end{itemize}
\end{frame}

%% -------------------------------------------------------------------------- %%

\begin{frame}{Дорога к релизу}
  \begin{itemize}
    \item \soutt{Выбросить весь код и написать заново.}
    \item Создать новый проект и вдумчиво вручную
          перенести в него удачные части кода.
    \item Неудачные --- переписать с нуля.
  \end{itemize}
\end{frame}

% TODO: Cull down more slides!
% TODO: Upload to the flash drive!

%% ========================================================================== %%

\section{Вопросы?}

% Using [plain] to hide frame number.
\begin{frame}[plain]{Questions?}

\begin{center}
\Huge{@agladysh}
\end{center}

\begin{center}
\Large{ag@logiceditor.com}
\end{center}

\begin{center}
\href{http://meetup.com/Lua-in-Moscow}{meetup.com/Lua-in-Moscow}
\\
Come to our Lua conference in March 5 in Moscow!
\end{center}

\end{frame}

%% -------------------------------------------------------------------------- %%

\end{document}
