% Build with xetex.
\documentclass[aspectratio=169,handout,bigger]{beamer}

%% -------------------------------------------------------------------------- %%

\usepackage{xunicode}
\usepackage{xltxtra}
\usepackage{graphicx}

\usepackage{color}

\usepackage{setspace}
\usepackage{ragged2e}

\usepackage[normalem]{ulem}

\newcommand{\soutt}[1]{%
    \renewcommand{\ULthickness}{1pt}%
       \sout{#1}%
    \renewcommand{\ULthickness}{.4pt}% Resetting to ulem default
}

\usepackage{minted}

\usepackage{tikz}
\usetikzlibrary{shapes,arrows,chains,calc,fit,matrix}

%% -------------------------------------------------------------------------- %%

\usepackage{polyglossia}
\setmainlanguage{russian}
\setotherlanguage{english}

% NB: To get MS fonts for OS X:
%
% 1) Install MS Open XML Converter.
% 2) Install OpenXML_all_fonts.pkg from inside of Converter's mpkg.
%
% http://www.labnol.org/software/tutorials/\
% free-download-calibri-font-on-mac/3684/

\setmainfont{Calibri}
\setsansfont{Calibri}
\setmonofont{Consolas}

%% ========================================================================== %%

\useinnertheme{circles}

\setbeamertemplate{navigation symbols}{}
\setbeamertemplate{section in toc}[sections numbered]

\definecolor{chart11}{RGB}{0, 0, 0}
\setbeamercolor{title}{fg=chart11}
\setbeamercolor{author}{fg=chart11}
\setbeamercolor{frametitle}{fg=chart11}
\setbeamercolor{itemize item}{fg=chart11}
\setbeamercolor{itemize subitem}{fg=chart11}
\setbeamercolor{itemize subsubitem}{fg=chart11}
\setbeamercolor{section in toc}{fg=chart11}

\setbeamerfont{title}{series=\bfseries,parent=structure}

%% ========================================================================== %%

\title{Rapid backend prototyping for a geolocation-based mobile game}
\subtitle{With OpenResty, Redis and Docker}
\author{\includegraphics[height=.4\textheight]{logo}}
\institute{Alexander Gladysh <ag@logiceditor.com>\newline@agladysh}
\date{FOSDEM 2017}

%% ========================================================================== %%

\begin{document}

% Using [plain] to avoid frame number on the title page.
\begin{frame}[plain]
 \titlepage
\end{frame}

%% ========================================================================== %%

\begin{frame}{Talk plan}

\tableofcontents

\end{frame}

%% ========================================================================== %%

\section*{}

\begin{frame}{About me}

\begin{itemize}
\item Developing software since 2002
\item Most of the time in gamedev
\item Beyond that: high-load internet solutions, enterprise software etc. etc.
\item Organizer of
      \href{http://meetup.com/Lua-in-Moscow}{meetup.com/Lua-in-Moscow}. \\
      Come to our Lua-related conference in Moscow on March 5th!
\end{itemize}

\end{frame}

%% ========================================================================== %%

\section{The Case}

%% ========================================================================== %%

\begin{frame}{Mobile games with geolocation}
  \centering{\includegraphics[height=.75\textheight]{pokemongo}}
\end{frame}

%% -------------------------------------------------------------------------- %%

\begin{frame}{The Goals}
  \begin{itemize}
    \item To try out a number of approaches to the gameplay,
          to generate new ideas,
          to figure out what is fun and what is not.
    \item To get as cheaply as possible the framework that would allow
          to iterate over gameplay variants as fast as possible.
    \item To figure out technical limitations of the genre in practice.
  \end{itemize}
\end{frame}

%% -------------------------------------------------------------------------- %%

\begin{frame}{Results}
  A geolocation game server-side prototype along with a rudimentary
  client-side was developed in less than 100 man-hours (2 calendar months)
  \vspace*{1em}\par
  We're rapidly iterating over the gameplay options
  and develop the technology part of the project.
\end{frame}

%% -------------------------------------------------------------------------- %%

\begin{frame}{What is this talk about?}
  \begin{itemize}
    \item This talk is about the technology part of the project,
    \item not about game-design
    \item or monetization.
  \end{itemize}
  \vspace*{1em}\par
  It now is easier when ever to develop geolocation-based games.
  \vspace*{1em}\par
  I will show where to begin.
\end{frame}

%% ========================================================================== %%

\section{The Stack}

%% ========================================================================== %%

\begin{frame}{First prototype gameplay}
  \begin{itemize}
    \item The player is searching for
          the mobs placed on a map by walking around;
    \item Player has a set chance to catch the found mob.
    \item Caught mobs increase a stats counter in player characteristics.
    \item Caught mobs disappear from the map,
          but respawn at the same place after a set time.
    \item Admin users may add new mobs on the map.
  \end{itemize}
\end{frame}

%% -------------------------------------------------------------------------- %%

\begin{frame}{The Stack}
  \begin{itemize}
    \item Server:
    \begin{itemize}
      \item Redis,
      \item OpenResty,
      \item Docker.
    \end{itemize}
    \item Client:
    \begin{itemize}
      \item A single-page web application (in browser),
      \item HTML5.
    \end{itemize}
  \end{itemize}
\end{frame}

%% -------------------------------------------------------------------------- %%

\begin{frame}{Redis}
  \begin{itemize}
    \item A reliable, proven solution.
    \item Supports geoposition out of the box:
    \begin{itemize}
      \item \texttt{GEOADD key longitude latitude member}
      \item \texttt{GEORADIUS key longitude latitude radius m}
    \end{itemize}
    \item Useful set of primitives to store game objects in.
    \item "Stored procedures" in Lua.
  \end{itemize}
\end{frame}

%% -------------------------------------------------------------------------- %%

\begin{frame}{OpenResty}
  \begin{itemize}
    \item An nginx distributive supporting
          Lua, Redis and many other things out of the box.
    \item Very fast, rather friendly and well-maintained.
    \item Useful both for quick prototyping and production environment.
  \end{itemize}
\end{frame}

%% -------------------------------------------------------------------------- %%

\begin{frame}{Docker}
  \begin{itemize}
    \item Reproducible cross-platform development environment.
    \item Efficiently relieves the development environment pain set up.
    \item Can be quickly turned into a prototype of production environment
          (it is arguable if it is suitable for real production).
    \item Has to be updated to a sufficiently recent version.
  \end{itemize}
\end{frame}

%% -------------------------------------------------------------------------- %%

\begin{frame}{Browser, HTML5}
  \begin{itemize}
    \item Server-side during early stages of development is more important.
    \item Don't implement what can be imagined while play-testing,
          like augmented-reality combat and other frills.
    \item HTML5 is good to implement "quick and dirty" client.
    \item NB: Geolocation data access is limited, but sufficient.
  \end{itemize}
\end{frame}

%% ========================================================================== %%

\section{Docker}

%% ========================================================================== %%

\begin{frame}{Docker: how to install on Ubuntu}
  \begin{itemize}
    \item Traditionally, Ubuntu version is outdated.
    \item Don't do \texttt{wget | sh}.
    \item Install \texttt{docker} и \texttt{docker-machine}
          from docker's apt-repository (see manual).
    \item \texttt{docker-compose} is to be installed by \texttt{pip install}.
    \item Read the manual for the installation on other platforms.
  \end{itemize}
\end{frame}

%% -------------------------------------------------------------------------- %%

\begin{frame}{Docker on the developer's machine}

\begin{center}
\begin{tikzpicture}
  every join/.style={line},    % Default linetype for connecting boxes
  scale=1, transform shape
]

\tikzset{
  basebase/.style={
    align=center, minimum height=2em, minimum width=9em, color=chart11,
    inner sep=.5em, node distance=3em and 11em
  },
  rect/.style={
    basebase, draw, on grid
  },
  coord/.style={coordinate},
  line/.style={->, draw, chart11},
}

\node[rect, rounded corners]                     (client)    {Client};
\node[rect, rounded corners, below of=client]    (docker)    {localhost:8080};
\node[rect, below of=docker]    (openresty) {OpenResty};
\node[rect, below of=openresty] (redis)     {Redis};

\draw [line] (client)    -- (docker);
\draw [line] (docker)    -- (openresty);
\draw [line] (openresty) -- (redis);

\end{tikzpicture}
\end{center}

\end{frame}

%% -------------------------------------------------------------------------- %%

\begin{frame}[fragile]{docker-compose.yml for development: Redis}
\begin{minted}{text}
version: "2"
services:
  redis:
    image: redis
    volumes:
      - ./redis:/data
    command: redis-server --appendonly yes
  openresty: <...>
\end{minted}
\end{frame}

%% -------------------------------------------------------------------------- %%

\begin{frame}[fragile]{OpenResty: interesting parts of nginx.conf (reduced)}
\begin{minted}{nginx}
error_log logs/error.log notice;
http {
  include resolvers.conf;
  lua_package_path "$prefix/lualib/?.lua;;";
  lua_code_cache off; # TODO: Enable on production!
  server {
    listen 8080;
    include mime.types;
    default_type application/json;
    location / { index index.html; root static/; }
    location = /api/v1/ { content_by_lua_file 'api/index.lua'; }
  }
}
\end{minted}
\end{frame}

%% -------------------------------------------------------------------------- %%

\begin{frame}[fragile]{OpenResty: Dockerfile}
\begin{minted}{docker}
FROM openresty/openresty
COPY bin/entrypoint.sh /usr/local/bin/openresty-entrypoint.sh
COPY nginx/conf /usr/local/openresty/nginx/conf
COPY nginx/lualib /usr/loca/openresty/nginx/lualib
COPY nginx/lua /usr/loca/openresty/nginx/lua
COPY nginx/static /usr/loca/openresty/nginx/static
ENTRYPOINT /usr/local/bin/openresty-entrypoint.sh
\end{minted}
\end{frame}

%% -------------------------------------------------------------------------- %%

\begin{frame}[fragile]{OpenResty: entrypoint.sh}
\begin{minted}{bash}
#!/bin/sh
grep nameserver /etc/resolv.conf \
  | awk '{print  "resolver " $2 ";"}' \
  > /usr/local/openresty/nginx/conf/resolvers.conf
/usr/local/openresty/bin/openresty -g 'daemon off;' "$@"
\end{minted}
\end{frame}

%% -------------------------------------------------------------------------- %%

\begin{frame}[fragile]{docker-compose.yml for development: OpenResty}
\begin{minted}{text}
  <...>
  openresty:
    build: .
    ports:
      - "8080:8080"
    volumes:
      - ./nginx/lualib:/usr/local/openresty/nginx/lualib:ro
      - ./nginx/api:/usr/local/openresty/nginx/api:ro
      - ./nginx/static:/usr/local/openresty/nginx/static:ro
    links:
      - redis
\end{minted}
\end{frame}

%% ========================================================================== %%

\section{HTTP API}

%% ========================================================================== %%

\begin{frame}{API calls}
  \begin{itemize}
    \item Player calls:
      \begin{itemize}
        \item \texttt{/} get game world state,
        \item \texttt{/go/:go-id/} get game object state,
        \item \texttt{/go/:go-id/act/:action-id} do action.
      \end{itemize}
    \item System calls:
    \begin{itemize}
      \item \texttt{/register} create a player,
      \item \texttt{/reset} factory-reset the DB,
      \item \texttt{/patch} upgrade the DP to the current version.
    \end{itemize}
    \item NB: We're not implementing a back-office UI to save effort,
          but using in-game mechanics for the in-game management.
  \end{itemize}
\end{frame}

%% ========================================================================== %%

\section{Game World Architecture}

%% ========================================================================== %%

\begin{frame}{Game Object}
  \begin{itemize}
    \item From the server's point of view, game world consists of game objects.
    \item A game object has numeric characteristics and a list of actions.
    \item A game object may have coordinates.
    \item Position-less game objects must either belong to other objects
          or be used as their prototypes.
  \end{itemize}
\end{frame}

%% -------------------------------------------------------------------------- %%

\begin{frame}{Prototype chain}
  \begin{itemize}
    \item A game object might have the prototype.
    \item A prototype game object might also have the prototype.
    \item A game object derives its characteristics and actions
          from its prototypes.
  \end{itemize}
\end{frame}

%% -------------------------------------------------------------------------- %%

\begin{frame}{Characteristics}
  \begin{itemize}
    \item A characterisic is a named numeric property of a game object.
    \item If a game object does not have given characteristic set,
          its value is taken from the closest prototype in chain
          that has it set. If no such prototype found, the value is 0.
  \end{itemize}
\end{frame}

%% -------------------------------------------------------------------------- %%

\begin{frame}{Actions}
  \begin{itemize}
    \item Game object action is a key from the Lua table of action handlers.
    \item A player may initiate action from a game object
          if he has sufficient permissions.
  \end{itemize}
\end{frame}

%% -------------------------------------------------------------------------- %%

\begin{frame}[fragile]{A Green Toad Mob}
\begin{minted}{lua}
{
  id = 'proto.mob.collectable';
  chrs = { respawn_dt = 10 * 60 };
  actions = { 'mob.collect' };
};
{
  id = 'proto.mob.toad.green';
  proto_id = 'proto.mob.collectable';
  chrs = { escape_chance = 0.25 };
};
{
  id = 'fa2eb7bca46c11e6be447831c1cebc82';
  proto_id = 'proto.mob.toad.green';
  geo = { lat = 55.7558, lon = 37.6173 };
};
\end{minted}
\end{frame}

%% -------------------------------------------------------------------------- %%

\begin{frame}[fragile]{Action: Catch the Mob}
\begin{minted}{lua}
ACTIONS['mob.collect'] = function(target, initiator)
  if
    math.random() * initiator.chrs.collect_skill >
    target.chrs.escape_chance
  then
    -- Inc number of catches for this mob type
    go_inc_chr(initiator.id, target.proto_id, 1)
    go_schedule_action_initiation( -- Schedule respawn
      target.chrs.respawn_dt, 'mob.spawn',
      { proto_id = target.proto_id, pos = target.pos },
      initiator.id
    )
    go_remove(target.id) -- Mob is caught, remove
  end
end
\end{minted}
\end{frame}

%% -------------------------------------------------------------------------- %%

\begin{frame}[fragile]{Delayed Action Execution}
\begin{minted}{lua}
local timestamp = os.time()
local action_ids = redis:zrangebyscore('da', '-inf', timestamp)
for i = 1, #action_ids do
  -- Execute action_ids[i] action
end
redis:zremrangebyscore('da', '-inf', timestamp)
\end{minted}
\end{frame}

%% -------------------------------------------------------------------------- %%

\begin{frame}[fragile]{Player}
\begin{minted}{lua}
{
  id = 'proto.user';
  chrs = { vision = 100, reach = 50 };
};
{
  id = 'user.1';
  geo = { lat = 55.7558, lon = 37.6173 };
  chrs = { collect_skill = 0.5 };
};
\end{minted}
\end{frame}

%% -------------------------------------------------------------------------- %%

\begin{frame}[fragile]{Item: Admin Hat}
\begin{minted}{lua}
{
  id = 'proto.item.wearable';
  actions = {
    ['item.don'] = { enabled = true };
    ['item.doff'] = { enabled = false };
  };
}
{
  id = 'proto.item.admin-hat';
  proto_id = 'proto.item.wearable';
  grants = { 'user.admin' };
  chrs = { collect_skill = 0.25 };
};
\end{minted}
\end{frame}

%% -------------------------------------------------------------------------- %%

\begin{frame}[fragile]{Let's issue Admin Hat to the player}
\begin{minted}{lua}
local hat = go_new('proto.item.admin-hat')

assert(go_get('user.1').stored[1] == nil)

go_store('user.1', hat.id)

assert(go_get('user.1').stored[1] == hat.id)
\end{minted}
\end{frame}

%% -------------------------------------------------------------------------- %%

\begin{frame}{Storing game objects ("storage")}
  \begin{itemize}
    \item A game object ("container") might "store" other game objects.
    \item Stored objects are not "visible" outside of their container.
    \item Player (a game object too) may initiate actions
          from the game objects he directly stores.
    \item Stored game objects don't affect characteristics of their containers.
  \end{itemize}
\end{frame}

%% -------------------------------------------------------------------------- %%

\begin{frame}[fragile]{Admin Hat actions}
\begin{minted}{lua}
ACTIONS['item.don'] = function(target, initiator)
  go_unstore(initiator.id, target.id)
  go_attach(initiator.id, target.id)
  go_disable_action(target.id, 'item.don')
  go_enable_action(target.id, 'item.doff')
end

ACTIONS['item.doff'] = function(target, initiator)
  go_attach(initiator.id, target.id)
  go_store(initiator.id, target.id)
  go_disable_action(target.id, 'item.doff')
  go_enable_action(target.id, 'item.don')
end
\end{minted}
\end{frame}

%% -------------------------------------------------------------------------- %%

\begin{frame}{Attaching / wearing objects}
  \begin{itemize}
    \item A game object may be attached to another game object.
    \item Attached objects are visible outside of the parent object.
    \item Player (a game object too) may initiate actions
          from the game objects that are directly attached to it.
    \item Attached game object characteristics are added to the parent object
          characteristics.
  \end{itemize}
\end{frame}

%% -------------------------------------------------------------------------- %%

\begin{frame}[fragile]{Item: Green Toad Spawner}
\begin{minted}{lua}
{
  id = 'proto.item.spawner.toad.green';
  actions = {
    ['mob.spawn'] = {
      requires = { 'user.admin' };
      param = { proto_id = 'proto.mob.toad.green' };
    };
  };
};
\end{minted}
\end{frame}

%% -------------------------------------------------------------------------- %%

\begin{frame}{Action permissions}
  \begin{itemize}
    \item An action is available only if \texttt{grants} list of the player
          contains all ids from \texttt{requires} list of the action.
    \item Items, attached to the player add their \texttt{grants} to it.
  \end{itemize}
\end{frame}

%% ========================================================================== %%

\section{Demo}

%% ========================================================================== %%

\begin{frame}{Demo}
  \begin{itemize}
    \item Current version of the application can be found here
          \href{https://geo.logiceditor.com/}{geo.logiceditor.com}.
    \item Sources are linked to at the same page.
    \item The very first client is always \texttt{curl}.
          You can try out the API using it by appending
          \texttt{/api/v1} to the URL of the application.
  \end{itemize}
\end{frame}

%% ========================================================================== %%

\section{Client}

%% ========================================================================== %%

\begin{frame}{How does the client work?}
  \begin{itemize}
    \item Geolocation and googlemaps are initalized.
    \item Current geoposition is sent to the server.
    \item Server returns a list of visible objects with their available actions.
    \item Markers are placed on the map for each object,
          and object details are rendered below the map by the HTML layout.
    \item Client awaits action initiation or coordinate change.
  \end{itemize}
  \vspace*{1em}\par
  NB:
  \begin{itemize}
    \item Toad names are generated with change.js using object uuid as a seed.
  \end{itemize}
\end{frame}

%% -------------------------------------------------------------------------- %%

\begin{frame}{HTML5 Geolocation}
  \begin{itemize}
    \item \texttt{navigator.geolocation.watchPosition(callback, options)}.
    \item Use developer panel tab Sensors to debug geolocation in Chrome.
    \item Chrome disables geolocation for HTTP sites (except localhost).
          Use HTTPS (e.g. with a free Let's Encrypt certificate).
  \end{itemize}
\end{frame}

%% -------------------------------------------------------------------------- %%

\begin{frame}[fragile]{Google Maps}
\begin{minted}{js}
new google.maps.Map(assert(document.getElementById('map')), {
  center: new google.maps.LatLng(pos.lat, pos.lon),
  zoom: 18,
  mapTypeId: google.maps.MapTypeId.ROADMAP,
  disableDefaultUI: true,
  disableDoubleClickZoom: true,
  draggable: false,
  scrollwheel: false,
  styles: [ { featureType: "poi",
    stylers: [ { visibility: "off" } ]
  } ]
});
\end{minted}
\end{frame}

%% ========================================================================== %%

\section{Conclusion}

%% ========================================================================== %%

\begin{frame}{Problems}
  \begin{itemize}
    \item Noisy geoposition data.
    \item Extremely low precision of the geolocation in the buildings.
    \item ...
  \end{itemize}
\end{frame}

%% -------------------------------------------------------------------------- %%

\begin{frame}{\soutt{Problems} Technical limitations}
  \begin{itemize}
    \item Noisy geoposition data.
    \item Extremely low precision of the geolocation in the buildings.
    \item ...
  \end{itemize}

  There are no problems. There are technical limitations you have to
  consider while designing gameplay. Some of the limitations may be
  resolved technologically. If you need to spend time of it --- is one
  of the questions for the preproduction stage.
\end{frame}

%% -------------------------------------------------------------------------- %%

\begin{frame}{Missing mechanics in the current implementation}
  \begin{itemize}
    \item Event system --- the largest.
    \item Lots of small functions, like pedometer.
  \end{itemize}
\end{frame}

%% -------------------------------------------------------------------------- %%

\begin{frame}{A Way to Release}
  \begin{itemize}
    \item \soutt{Throw away the code and write it from scratch.}
    \item Create a new project and thoughtfully and manually
          move good parts of the prototype there.
    \item Rewrite from scratch bad parts.
  \end{itemize}
\end{frame}

% TODO: Cull down more slides!
% TODO: Announce the March 5th conference on the meetup page
% TODO: Upload to the flash drive!

%% ========================================================================== %%

\section{Questions?}

% Using [plain] to hide frame number.
\begin{frame}[plain]{Questions?}

\begin{center}
\Huge{@agladysh}
\end{center}

\begin{center}
\Large{ag@logiceditor.com}
\end{center}

\begin{center}
Game and code for this talk:
\href{https://geo.logiceditor.com/}{geo.logiceditor.com}
\end{center}

\begin{center}
\href{http://meetup.com/Lua-in-Moscow}{meetup.com/Lua-in-Moscow}
\\
Come to our Lua conference in March 5th in Moscow!
\end{center}

\end{frame}

%% -------------------------------------------------------------------------- %%

\end{document}
