% Build with xetex.
\documentclass[aspectratio=43,handout,bigger]{beamer}

%% -------------------------------------------------------------------------- %%

\usepackage{xunicode}
\usepackage{xltxtra}
\usepackage{graphicx}

\usepackage{color}

\usepackage{setspace}
\usepackage{ragged2e}

\usepackage[normalem]{ulem}

\usepackage{minted}

\usepackage{tikz}
\usetikzlibrary{shapes,arrows,chains,calc,fit,matrix}

%% -------------------------------------------------------------------------- %%

\usepackage{polyglossia}
\setmainlanguage{russian}
\setotherlanguage{english}

% NB: To get MS fonts for OS X:
%
% 1) Install MS Open XML Converter.
% 2) Install OpenXML_all_fonts.pkg from inside of Converter's mpkg.
%
% http://www.labnol.org/software/tutorials/\
% free-download-calibri-font-on-mac/3684/

\setmainfont{Calibri}
\setsansfont{Calibri}
\setmonofont{Consolas}

%% -------------------------------------------------------------------------- %%

\mode<presentation>{
  \usetheme{default}
}

\useinnertheme{circles}

\setbeamertemplate{navigation symbols}{}
\setbeamertemplate{section in toc}[sections numbered]

\definecolor{chart11}{RGB}{0, 0, 0}
\setbeamercolor{title}{fg=chart11}
\setbeamercolor{author}{fg=chart11}
\setbeamercolor{frametitle}{fg=chart11}
\setbeamercolor{itemize item}{fg=chart11}
\setbeamercolor{itemize subitem}{fg=chart11}
\setbeamercolor{itemize subsubitem}{fg=chart11}
\setbeamercolor{section in toc}{fg=chart11}

%% -------------------------------------------------------------------------- %%

\defbeamertemplate{footline}{left page number}
{%
  \hfill
  \usebeamercolor[fg]{page number in head/foot}%
  \usebeamerfont{page number in head/foot}%
  \insertpagenumber\,/\,\insertpresentationendpage%
  \hspace*{1em}
  \vskip1em%
}
\setbeamertemplate{footline}[left page number]

%% ========================================================================== %%

\title{\includegraphics[height=.15\textheight]{logo}}
\author{Quick functional UI sketches\\with Lua templates and mermaid.js}
\institute{Alexander Gladysh\\@agladysh}
\date{FOSDEM 2017}

%% ========================================================================== %%

\begin{document}

% Using [plain] to avoid frame number on the title page.
\begin{frame}[plain]
 \titlepage
\end{frame}

%% ========================================================================== %%

\begin{frame}{Talk plan}

\tableofcontents

\end{frame}

%% ========================================================================== %%

\section*{}

\begin{frame}{About me}

\begin{itemize}
\item Programmer background
\item Mainly doing management work now
\item In löve with Lua since 2005
\end{itemize}

\end{frame}

%% ========================================================================== %%

\section{The Case}

%% ========================================================================== %%

\begin{frame}{The Case}
  \begin{itemize}
    \item A huge professional enterprise application
    \item being converted from 20-year-old windows app
    \item to a modern SPA web-app.
  \end{itemize}

  TODO: Screenshots
\end{frame}

%% -------------------------------------------------------------------------- %%

\begin{frame}{The product is huge}
  Sufficient expertise can only be found on a team level:

  \begin{itemize}
    \item Technology experts don't have product-level vision
    \item PO and PM don't draw (and lack deep insight on the tech)
    \item Designer does not have the technology expertise
  \end{itemize}
\end{frame}

%% -------------------------------------------------------------------------- %%

\begin{frame}{The UI design and development process}
  For each "screen" in the application:

  \begin{itemize}
    \item Concept
    \item Functional sketches and (sometimes) interactive studies
    \item Design sketches
    \item Layout implementation
    \item Business logic implementation
  \end{itemize}

  TODO: Screenshots
\end{frame}

%% -------------------------------------------------------------------------- %%

\begin{frame}{A functional sketch}
  What is on the screen, how does it WORK?

  TODO: Screenshot
\end{frame}

%% -------------------------------------------------------------------------- %%

\begin{frame}{A design sketch}
  How does it LOOK?

  TODO: Screenshot
\end{frame}

%% ========================================================================== %%

% TODO
% -- Alternative approaches: should not chafe and hamper design process
% -- main goal is to iterate over designs rapidly
% -- I don't have skills to do that in Photoshop / InkScape / Sketch / balsamiq
% -- / Visio / whatever drawing / UI design software. I tend to focus
% -- on the visual, not functional and visual is irrelevant on this stage.
% -- I work fastest when I use keyboard and don't touch the mouse
% -- Teamwork on the source level doesn't matter, I'll be the only person
% -- who produces the images (although they are discussed by a group).
%
% -- Another alternative is text descriptions, which are impossible to iterate
% -- fast, and quickly become huge unmanageable entangled mess of cruft.
% -- Excel is a bit better (for some screens), but it lacks a bit in visual
% -- representation means.
%
% -- HTML works best for me, good old HTML4 from pre-js era.
%
% -- Tools: mermaid (MIT license) (mermaid cli tool, svg, phantomjs)
% -- Show the UI screen flow diagram first
% -- Then show how to embed some HTML
% -- Then explain the visual language (arrows and stuff)
% -- Then... TODO
% -- Lua (MIT license) + Our template engine (MIT license) (+a remark on LaTeX)
% -- How the engine works TODO
%
% -- Show how the tool is being run
% -- Show tricks from prologue
% -- Show how the production templates look
%
% -- How to approach the design (keep it as simple as possible,
% -- some copy-paste is ok, generalize code/stuff when it chafes,
% -- but not much earlier, to be able to iterate fast, you need to be able
% -- to throw away stuff, to be able to throw away stuff easily, you
% -- have to invest as little as possible in it; important to remember
% -- that it is simply an image, not an interactive page)
%
% -- Problems
% -- HTML4-era debugging for layout.
% -- Lack of debugging tools for the template engine (easily added)
% -- Some lack of expressive power (easily added if needed)
%
% -- How much effort went into it? (two days to code the core,
% -- maybe three more days overall to work on the core along the 6 months
% -- it is there; about 40 screens designed and many more to come)
%
% -- Was it worth it? Yes. Should you use it? If you're so inclined.
% -- But look at other tools too, this one works for *me*.
%
% -- A link to source code

%% ========================================================================== %%

\section{Questions?}

%% ========================================================================== %%

% Using [plain] to hide frame number.
\begin{frame}[plain]{Questions?}

\begin{center}
\Huge{@agladysh}
\end{center}

\begin{center}
\Large{agladysh@gmail.com}
\end{center}

\end{frame}

%% ========================================================================== %%

\end{document}
