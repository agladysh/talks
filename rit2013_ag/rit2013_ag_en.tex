% Build with xetex.
\documentclass[aspectratio=169,handout,bigger]{beamer}

%% -------------------------------------------------------------------------- %%

\usepackage{xunicode}
\usepackage{xltxtra}
\usepackage{graphicx}

\usepackage{color}

\usepackage{setspace}
\usepackage{ragged2e}

%% -------------------------------------------------------------------------- %%

\usepackage{polyglossia}
\setmainlanguage{english}
\setotherlanguage{russian}

% NB: To get MS fonts for OS X:
%
% 1) Install MS Open XML Converter.
% 2) Install OpenXML_all_fonts.pkg from inside of Converter's mpkg.
%
% http://www.labnol.org/software/tutorials/\
% free-download-calibri-font-on-mac/3684/

\setmainfont{Calibri}
\setsansfont{Calibri}
\setmonofont{Consolas}

%% -------------------------------------------------------------------------- %%

\mode<presentation>{
  \usetheme{default}
}

\useinnertheme{circles}

\setbeamertemplate{navigation symbols}{}
\setbeamertemplate{section in toc}[sections numbered]

\definecolor{chart11}{RGB}{197, 0, 11}
\setbeamercolor{title}{fg=chart11}
\setbeamercolor{author}{fg=chart11}
\setbeamercolor{frametitle}{fg=chart11}
\setbeamercolor{itemize item}{fg=chart11}
\setbeamercolor{itemize subitem}{fg=chart11}
\setbeamercolor{itemize subsubitem}{fg=chart11}
\setbeamercolor{section in toc}{fg=chart11}

\addtobeamertemplate{frametitle}{
  \ifx\insertsubsection\empty
    \let\insertframetitle\insertsectionhead
  \else
    \let\insertframetitle\insertsubsectionhead
  \fi
}{}

\addtobeamertemplate{frametitle}{
  \ifx\insertsubsection\empty
  \else
    \let\insertframesubtitle\insertsectionhead
  \fi
}{}

\makeatletter
  \CheckCommand*\beamer@checkframetitle{%
    \@ifnextchar\bgroup\beamer@inlineframetitle{}}
  \renewcommand*\beamer@checkframetitle{%
    \global\let\beamer@frametitle\relax\@ifnextchar%
    \bgroup\beamer@inlineframetitle{}}
\makeatother

%% ========================================================================== %%

\title{Using LuaJIT in mid-load web-projects}
\author{Alexander Gladysh\\CTO, LogicEditor}
\date{}

%% ========================================================================== %%

\begin{document}

\usebackgroundtemplate{
  \includegraphics[width=\paperwidth,height=\paperheight]{back_title}
}

\maketitle

%% ========================================================================== %%

\usebackgroundtemplate{
  \includegraphics[width=\paperwidth,height=\paperheight]{back}
}

\begin{frame}{Contents}

\tableofcontents

\end{frame}

%% ========================================================================== %%

%\section{О Lua и LuaJIT}

%% -------------------------------------------------------------------------- %%

% TODO: Hack!
\section*{}
\subsection*{On Lua and LuaJIT}

\begin{frame}{}
  Lua:
  \begin{itemize}
    \item powerful,
    \item fast,
    \item lightweight,
    \item extensible,
    \item embeddable
  \end{itemize}
  scripting programming language.
\end{frame}

%% -------------------------------------------------------------------------- %%

\section{On Lua and LuaJIT}
\subsection*{Language}

\begin{frame}
  \begin{itemize}
    \item Origins.
    \item Popularity growth.
    \item Where is Lua used?
    \item LuaRocks.
  \end{itemize}
\end{frame}

%% -------------------------------------------------------------------------- %%

\subsection*{Popular Dialects}

\begin{frame}
  \begin{itemize}
    \item Lua 5.1 vs. Lua 5.2,
    \item LuaJIT 2.0,
    \item Metalua.
  \end{itemize}
\end{frame}

%% -------------------------------------------------------------------------- %%

\subsection*{LuaJIT 2.0}

\begin{frame}
  \begin{itemize}
    \item JIT, FFI, performance.
    \item Limitations on x86\_64.
    \item LuaJIT vs. Lua 5.2.
  \end{itemize}
\end{frame}

%% -------------------------------------------------------------------------- %%

\subsection*{Why Lua?}

\begin{frame}
  Historically: We're coming from computer games industry, where Lua "rules the world".
  \\~\\
  Pragmatically:
  \begin{itemize}
    \item Works fast!
    \item Pleasant to code!
    \item Easy to learn!
  \end{itemize}
\end{frame}

%% -------------------------------------------------------------------------- %%

\subsection*{Where to get programmers?}

\begin{frame}
  \begin{center}
    Reeducation.
  \end{center}
\end{frame}

%% -------------------------------------------------------------------------- %%

\subsection*{Main problems learning Lua}

\begin{frame}
  \begin{itemize}
    \item Mindless tinkering with the language.
    \begin{itemize}
      \item NIH-syndrome. Lua is to easy to tinker with.
      \item Diverging from the mainstream. Costs and benefits.
    \end{itemize}
    \item Language idiosyncrasies:
    \begin{itemize}
      \item Global variables by default.
      \item Arrays are indexed from 1.
      \item Size of array with nil element is not defined.
      \item Everything that is not nil or false — true (0 too).
    \end{itemize}
  \end{itemize}
\end{frame}

%% -------------------------------------------------------------------------- %%

\subsection*{Most important!}

\begin{frame}
  \centering When you code in Lua — code in Lua!
\end{frame}

%% -------------------------------------------------------------------------- %%

\subsection*{Lua / LuaJIT place in your stack}

\begin{frame}
  First and foremost:

  \begin{itemize}
    \item User-configurable business-logic.
    \item Code that would otherwise be written in C/C++/OCaml.
  \end{itemize}
\end{frame}

%% ========================================================================== %%

\section{Our stack}

%% -------------------------------------------------------------------------- %%

\subsection*{Frameworks to build webservices with Lua}

\begin{frame}
  Some of the popular ones:
  \begin{itemize}
    \item Kepler/WSAPI
    \item OpenResty
    \item Luvit
  \end{itemize}

  We have a "bicycle", built on WSAPI.
\end{frame}

%% -------------------------------------------------------------------------- %%

\subsection*{What web-problems are we solving with Lua?}

\begin{frame}
  \begin{itemize}
    \item Browser and social games.
    \item Ad networks.
    \item Other web-services, mobile games etc.
  \end{itemize}
\end{frame}

%% -------------------------------------------------------------------------- %%

\subsection*{Hardware}

\begin{frame}
  \begin{itemize}
    \item Linode
    \item Hetzner EX6
  \end{itemize}
\end{frame}

%% -------------------------------------------------------------------------- %%

\subsection*{OS}

\begin{frame}
  \begin{itemize}
    \item Xen XCP on top of Ubuntu Sever.
    \item domU on Ubuntu Server:
    \begin{itemize}
      \item HTTP frontends (nginx).
      \item HTTP backends (32-bit).
      \item Workers (32-bit).
      \item DB (Redis, MySQL).
      \item Aux (Bind, nginx-based config-server, deployment, monitoring etc.).
    \end{itemize}
  \end{itemize}
\end{frame}

%% -------------------------------------------------------------------------- %%

\subsection*{Backends}

\begin{frame}
  \begin{itemize}
    \item nginx
    \item spawn-fcgi + multiwatch
    \item LuaJIT 2.0
    \item FCGI/WSAPI
    \item Application code
  \end{itemize}
\end{frame}

%% -------------------------------------------------------------------------- %%

\subsection*{Workers}

\begin{frame}
  \begin{itemize}
    \item runit
    \item LuaJIT 2.0
    \item Application code
  \end{itemize}
\end{frame}

%% -------------------------------------------------------------------------- %%

\subsection*{IPC}

\begin{frame}
  \begin{itemize}
    \item ØMQ
    \item Tasks:
    \begin{itemize}
      \item Replacement for broken signals.
      \item In-process cache reset.
      \item (Workers get tasks via Redis.)
    \end{itemize}
  \end{itemize}
\end{frame}

%% -------------------------------------------------------------------------- %%

\subsection*{System tuning}

\begin{frame}
  \begin{itemize}
    \item OS
    \begin{itemize}
      \item \href{http://bit.ly/kernel-magic}{bit.ly/kernel-magic} (for frontends and backends)
    \end{itemize}
    \item Redis
    \begin{itemize}
      \item I/O Scheduler: noop on guests, deadline on host
    \end{itemize}
    \item nginx
    \begin{itemize}
      \item worker\_rlimit\_nofile
    \end{itemize}
  \end{itemize}
\end{frame}

%% -------------------------------------------------------------------------- %%

\subsection*{"DevOps"}

\begin{frame}
  \begin{itemize}
    \item Deployment
    \item High Availability
    \item Monitoring
  \end{itemize}
\end{frame}

%% -------------------------------------------------------------------------- %%

\subsection*{Main libraries}

\begin{frame}
  \begin{itemize}
    \item lua-nucleo, lua-aplicado (better — Penlight, telescope)
    \item slnunicode
    \item luatexts, luajson
    \item luasocket, luaposix (better — ljsyscall)
    \item WSAPI
    \item lua-zmq
    \item lua-hiredis (better — ljffi-hiredis)
    \item luasql-mysql
  \end{itemize}
\end{frame}

%% -------------------------------------------------------------------------- %%

\subsection*{DSL and code generation}

\begin{frame}
  \begin{itemize}
    \item HTTP request handlers.
    \begin{itemize}
      \item Code (partially statically validated).
      \item Docs.
      \item (Planned) Smoke-tests.
    \end{itemize}
    \item SQL schema.
    \begin{itemize}
      \item "ORM" wrapper code.
      \item DB schema patches.
      \item Auto-backoffice.
      \item Docs.
    \end{itemize}
    \item \href{http://bit.ly/lua-dsl-talk}{bit.ly/lua-dsl-talk}
    \item Common parts of a project are generated from text templates.
  \end{itemize}
\end{frame}

%% -------------------------------------------------------------------------- %%

\subsection*{Performance}

\begin{frame}
  \begin{itemize}
    \item About 160M synthetic hits per day \emph{per EX6-class server} in ad networks.
    \item About 8K simultaneous active users \emph{per EX6-class server} in online games.
  \end{itemize}
\end{frame}

%% ========================================================================== %%

\section{Pitfalls}

%% -------------------------------------------------------------------------- %%

\subsection*{What did we encounter?}

\begin{frame}
  In main:

  \begin{itemize}
    \item Couple "mysterious" problems, due to bugs in early LuaJIT2 betas (all fixed by now).
    \item Problems, caused by two versions of the same LuaRocks package installed in the system.
    \item Exploding Redis.
    \item Lousy Hetzner HDD reliability.
  \end{itemize}
\end{frame}

%% -------------------------------------------------------------------------- %%

\subsection*{Diagnostics, debugging and monitoring}

\begin{frame}
  \begin{itemize}
    % TODO!!!
    \item Partial static code validation.
    \item Runtime validation.
    \item Autotests.
    \item GC tuning and monitoring.
    \item Monitoring for request times, memory usage etc.
    \item Debugging with logs.
  \end{itemize}
\end{frame}

%% -------------------------------------------------------------------------- %%

\subsection*{Main unsolved problems}

\begin{frame}
  \begin{itemize}
    \item Long polling / Comet.
    \item OS signal handling.
    \item More efficient CPU usage with HTTP handlers.
    \item LuaRocks:
    \begin{itemize}
      \item Can install two versions of the same rock in the system.
      \item Can't upgrade a package.
    \end{itemize}
  \end{itemize}
\end{frame}

%% ========================================================================== %%

\section{Next generation stack}

%% -------------------------------------------------------------------------- %%

% TODO: Hack!
\section*{}
\subsection*{Next generation stack}

% TODO: Rethink this
\begin{frame}
  \begin{itemize}
    \item Unblocking API with coroutines, no callback. Get inspired by, or adapt OpenResty.
    \item Complete transition to LuaJIT FFI.
    \item Consider dropping LuaRocks.
    \item Drop FCGI, move to epoll and lua-http-parser.
    \item Simplify the architecture as much as possible. Drop the configuration server. More code generation!
    \item New DSL design.
  \end{itemize}
\end{frame}

%% ========================================================================== %%

\section{Want to know more?}

\begin{frame}
  \begin{center}
  \begin{minipage}{0.6\linewidth}
  \begin{itemize}
    \item[Official Site] \href{http://lua.org}{lua.org}, \href{http://luajit.org}{luajit.org}
    \item[Wiki] lua-users.org/wiki, \href{http://wiki.luajit.org}{wiki.luajit.org}
    \item[Mailing Lists] \href{http://lua.org/lua-l.html}{lua.org/lua-l.html}, \href{http://luajit.org/list.html}{luajit.org/list.html}
    \item[StackOverflow] \href{http://stackoverflow.com/questions/tagged/Lua}{stackoverflow.com/questions/tagged/Lua}
    \item[IRC] \#lua at \href{http://irc.freenode.net}{irc.freenode.net}
  \end{itemize}
  \end{minipage}
  \end{center}
\end{frame}

%% ========================================================================== %%

\section*{Questions?}

\begin{frame}

\begin{center}
\Huge{@agladysh}
\end{center}

\begin{center}
\Large{ag@logiceditor.com}
\end{center}

\begin{center}
\href{http://meetup.com/Lua-in-Moscow}{meetup.com/Lua-in-Moscow}
\end{center}

\end{frame}

%% -------------------------------------------------------------------------- %%

\end{document}
