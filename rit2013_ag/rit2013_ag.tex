% Build with xetex.
\documentclass[handout]{beamer}

\usepackage{xunicode}
\usepackage{xltxtra}
\usepackage{graphicx}

\usepackage{setspace}
\usepackage{ragged2e}

%% -------------------------------------------------------------------------- %%

\usepackage{polyglossia}
\setmainlanguage{russian}
\setotherlanguage{english}

\setmainfont{Liberation Serif}
\setsansfont{Liberation Sans}
\setmonofont{Liberation Mono}

%% -------------------------------------------------------------------------- %%

\mode<presentation>{
  \usetheme{default}
}

\setbeamertemplate{navigation symbols}{}
\setbeamertemplate{section in toc}[sections numbered]

\addtobeamertemplate{frametitle}{
  \ifx\insertsubsection\empty
    \let\insertframetitle\insertsectionhead
  \else
    \let\insertframetitle\insertsubsectionhead
  \fi
}{}

\addtobeamertemplate{frametitle}{
  \ifx\insertsubsection\empty
  \else
    \let\insertframesubtitle\insertsectionhead
  \fi
}{}

\makeatletter
  \CheckCommand*\beamer@checkframetitle{%
    \@ifnextchar\bgroup\beamer@inlineframetitle{}}
  \renewcommand*\beamer@checkframetitle{%
    \global\let\beamer@frametitle\relax\@ifnextchar%
    \bgroup\beamer@inlineframetitle{}}
\makeatother

%% -------------------------------------------------------------------------- %%

\title{Опыт работы с LuaJIT в нагруженных интернет-проектах}
\author{Александр ГЛАДЫШ\\LogicEditor, CTO\\ag@logiceditor.com}
\date{RIT++\\2013~г.}

%% -------------------------------------------------------------------------- %%

\begin{document}

\maketitle

%% -------------------------------------------------------------------------- %%

\begin{frame}{Содержание}

\tableofcontents

\end{frame}

%% -------------------------------------------------------------------------- %%

\setlength{\parskip}{\baselineskip}

%% -------------------------------------------------------------------------- %%

\section{Почему Lua?}

%% -------------------------------------------------------------------------- %%

\subsection*{Исторически}

\begin{frame}
  Мы вышли из игровой индустрии, где Lua правит миром.
\end{frame}

%% -------------------------------------------------------------------------- %%

\subsection*{Прагматически}

\begin{frame}
  \begin{itemize}
    \item Работает — быстро!
    \item Писать — удобно!
    \item Освоить — легко!
  \end{itemize}
\end{frame}

%% -------------------------------------------------------------------------- %%

\subsection*{Недостатки}

\begin{frame}
  \begin{itemize}
    \item В первую очередь — где искать людей?
    \item Основные проблемы при переучивании на Lua.
    \item Идеосинкразии языка.
    \item Пишите на Lua как на Lua!
  \end{itemize}
\end{frame}

%% ========================================================================== %%

\section{О Lua и LuaJIT}

\begin{frame}
  Очень кратко о языке Lua, его происхождении, особенностях и росте популярности в последние годы. Где используется язык? IDE и специализированные IDE. Мэйнстрим и самопальные диалекты, NIH-синдром и лёгкость доработки напильником. Цена и выгоды отхода от мэйнстрима. Lua 5.1 vs. Lua 5.2. Metalua.
\end{frame}

\begin{frame}
  LuaJIT 2.0: почти-мэйнстрим диалект Lua. JIT, FFI, производительность. Поддерживаемые платформы. Ограничения на 64-х битах. LuaJIT vs. Lua 5.2. Вкусности, планируемые для LuaJIT 2.1 и LuaJIT 3.
\end{frame}

\begin{frame}
  Встроенный vs. расширяемый язык (на самом деле и то и то)? Ситуация до LJ2 и после. Теперь можно больше не писать на C!
\end{frame}

\begin{frame}
  Раньше с кодом было туго, сейчас качественного готового кода на Lua много. LuaRocks.
\end{frame}

%% ========================================================================== %%

\section{Почему не писать всё PHP?}

\begin{frame}
Место для Lua / LuaJIT в вашем стэке? Как другие интернет-системы используют Lua? Как это делаем мы? С нами “всё ясно”, мы — хардкорщики из геймдева (на самом деле нет). Почему и где стоит начать применять технологии из этого доклада в существующих продакшен-системах?
\end{frame}

\begin{frame}
a) Настраиваемая пользователем логика.
\end{frame}

\begin{frame}
b) Отдельностоящие сервисы.
\end{frame}

\begin{frame}
c) Код, который иначе был бы написан на C/C++/OCaml.
\end{frame}

\begin{frame}
d) ...
\end{frame}

%% ========================================================================== %%

\section{О существующих решениях для реализации веб-сервисов на Lua}

\begin{frame}
Популярные:
a) Kepler/WSAPI — дёшево и сердито.
b) Luvit — модная бяка, навязывает чуждую мэйнстримному Lua нодовскую экосистему.
c) openresty — перспективный продукт китайской инженерной мысли.

Остальные — см. TODO
\end{frame}

%% ========================================================================== %%

\section{Наш нынешний стек}

\begin{frame}
a) Какие задачи мы решаем?
\end{frame}

\begin{frame}
b) На каком железе мы живём?
\end{frame}

\begin{frame}
c) Архитектура взаимодействия. XEN, Ubuntu (и её тюнинг), nginx (и его тюнинг), spawn-fcgi, multiwatch, LuaJIT 2, WSAPI, 0MQ, Redis (и его тюнинг). DNS-ы. Отдельностоящие сервисы. Почему так?
\end{frame}

\begin{frame}
d) Какие луашные библиотеки мы используем и почему? Годные альтернативы нашим историческим opensource-велосипедам (и какие из велосипедов — лучше альтернатив).
\end{frame}

\begin{frame}
e) Как сделано High Availability?
\end{frame}

\begin{frame}
f) Как устроен деплоймент?
\end{frame}

\begin{frame}
g) Как устроен мониторинг?
\end{frame}

\begin{frame}
h) Какие показатели по производительности? По стабильности?
\end{frame}

\begin{frame}
i) DSL для описания обработчиков запросов. Кодогенерация. Прочие рюшечки и сахар (бонус: DSL для описания SQL-данных с возможностью автогенерации продвинутого UI бэкофиса для этих данных).
\end{frame}

%% ========================================================================== %%

\section{Грабли}

\begin{frame}
a) Какие были основные проблемы? Как их решали? Несколько общих советов по отладке и оптимизации производительности при работе с Lua. Отладка отладчиком и по логам, оптимизация GC, какие параметры нужно мониторить. Профайлинг кода на LJ2. Автотесты.
\end{frame}

\begin{frame}
b) Какие проблемы не решены, и как с этим жить?
\end{frame}

\begin{frame}
i. Long polling / comet.
\end{frame}

\begin{frame}
ii. TODO
\end{frame}

%% ========================================================================== %%

\section{Каким мы видим стек следующего поколения?}

\begin{frame}
a) Ориентироваться на openresty, но не использовать его напрямую. Почему?
\end{frame}

\begin{frame}
b) Новая архитектура.
\end{frame}

\begin{frame}
i. Проще! Ещё проще!
\end{frame}

\begin{frame}
ii. Отказ от LuaRocks.
\end{frame}

\begin{frame}
iii. Полный переход на FFI.
\end{frame}

\begin{frame}
iv. Отказ от FCGI и WSAPI. Переход на epoll и библиотеку парсинга HTTP.
\end{frame}

\begin{frame}
v. Отказ от сервера конфигураций.
\end{frame}

\begin{frame}
vi Улучшенная High Availability.
\end{frame}

\begin{frame}
vii Неблокирующее API на корутинах, без коллбэков. Архитектура. Особенности реализации для основных сервисов (HTTP[S], Redis, MySQL/Postgres).
\end{frame}

\begin{frame}
viii. Новый дизайн DSL.
\end{frame}

\begin{frame}
ix. ...
\end{frame}

%% ========================================================================== %%

\section{Хотите знать больше?}

\begin{frame}
9. Рекомендуемые источники информации о Lua, LuaJIT и сопутствующих технологиях.
\end{frame}

%% ========================================================================== %%

\section{Вопросы?}

\begin{frame}

\begin{center}
ag@logiceditor.com
\end{center}

\end{frame}

%% -------------------------------------------------------------------------- %%

\end{document}
