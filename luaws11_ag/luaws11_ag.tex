\documentclass{beamer}

\usepackage{minted}
\usepackage[utf8]{inputenc}

\mode<presentation>{
  \usetheme{default}
}

\setbeamertemplate{navigation symbols}{}

%% -------------------------------------------------------------------------- %%

\title{Declarative Internal DSLs in Lua}
\subtitle{A Game-Changing Experience}
\author{Alexander Gladysh, ag@logiceditor.com\\\textbf{L}ogic\textbf{E}ditor.com CTO, co-founder}
\date{Lua Workshop 2011}

%% -------------------------------------------------------------------------- %%

\begin{document}

\maketitle

%% -------------------------------------------------------------------------- %%

\begin{frame}

\frametitle{Outline}

\tableofcontents

\end{frame}

%% -------------------------------------------------------------------------- %%

\section{Introduction}

%% -------------------------------------------------------------------------- %%

\begin{frame}[fragile]

\frametitle{Internal Declarative DSL in Lua}

\begin{minted}{lua}
namespace:method "title"
{
  data = "here";
}
\end{minted}

\end{frame}

%% -------------------------------------------------------------------------- %%

\begin{frame}[fragile]

\frametitle{...Without sugar}

\begin{minted}{lua}
_G["namespace"]:method(
    "title"
  ) ({
    ["data"] = "here";
  })
\end{minted}

\end{frame}

%% -------------------------------------------------------------------------- %%

\begin{frame}[fragile]

\frametitle{Naïve implementation}

\begin{minted}{lua}
namespace = { }

namespace.method = function(self, name)
   return function(data)
    -- ...do something
    -- ...with name and data
  end
end
\end{minted}

\end{frame}

%% -------------------------------------------------------------------------- %%

\section { Ad-hoc approach }

%% -------------------------------------------------------------------------- %%

\begin{frame}[fragile]

\frametitle{Hyphotetic UI description language}

\begin{minted}{lua}
ui:dialog "alert"
{
  ui:label "message";
  ui:button "OK"
  {
    on_click = function(self)
      self:close()
    end;
  };
}
\end{minted}

\end{frame}

%% -------------------------------------------------------------------------- %%

\begin{frame}[fragile]

\frametitle{UI description language "implementation", I}

\begin{minted}{lua}
function ui:label(title)
  return function(data)
    return GUI.Label:new(title, data)
  end
end

function ui:button(title)
  return function(data)
    return GUI.Button:new(title, data)
  end
end
\end{minted}

\end{frame}

%% -------------------------------------------------------------------------- %%

\begin{frame}[fragile]

\frametitle{UI description language "implementation", II}

\begin{minted}{lua}
function ui:dialog(title)
  return function(data)
    local dialog = GUI.Dialog:new(title)
    for i = 1, #data do
      dialog:add_child(data)
    end
    return dialog
  end
end
\end{minted}

\end{frame}

%% -------------------------------------------------------------------------- %%

\begin{frame}

\frametitle{Ad-hoc approach}

\begin{itemize}
\item[+] Easy to code simple stuff
\end{itemize}

But:

\begin{itemize}
\item[$-$] Easily grows out of control
\item[$-$] Difficult to reuse
\item[$-$] Hard to handle errors
\item[$-$] Hard to add new output targets
\end{itemize}

\end{frame}

%% -------------------------------------------------------------------------- %%

\section { More realistic case }

%% -------------------------------------------------------------------------- %%

\begin{frame}[fragile]

\frametitle{Practical example: HTTP handler}

\begin{minted}{lua}
api:url "/reverse"
{
  doc:description [[String reverser]]
  [[
    Takes a string and reverses it.
  ]]
  api:input { data:string "text" };
  api:output
  {
    data:node "result" { data:string "reversed" };
  };
  handler = function(param)
    return { reversed = param.text:reverse() }
  end;
}
\end{minted}

\end{frame}

%% -------------------------------------------------------------------------- %%

\begin{frame}

\frametitle{What do we want to get from that description?}

\begin{itemize}
\item \textbf{HTTP request handler itself}, with:
  \begin{itemize}
  \item Input validation
  \item Multi-format output serialization (JSON, XML, ...)
  \item Handler code static checks (globals, ...)
  \end{itemize}
\item \textbf{Documentation}
\item Low-level networking \textbf{client code}
\item Smoke \textbf{tests}
\end{itemize}

\end{frame}

%% -------------------------------------------------------------------------- %%

\begin{frame}[fragile]

\frametitle{Request handler: input validation}

\begin{minted}{lua}
INPUT_LOADERS["/reverse"] = function(checker, param)
  return
  {
    text = check.string(param, "string");
  }
end
\end{minted}

\end{frame}

%% -------------------------------------------------------------------------- %%

\begin{frame}[fragile]

\frametitle{Request handler: output serialization}

\begin{minted}{lua}
local build_formatter = function(fmt)
    return fmt:node("nil", "result")
    {
        fmt:attribute("reversed");
    }
end

OUTPUT["/reverse.xml"] = build_formatter(
      make_xml_formatter_builder()
  ):commit()

OUTPUT["/reverse.json"] = build_formatter(
      make_json_formatter_builder()
  ):commit()
\end{minted}

\end{frame}

%% -------------------------------------------------------------------------- %%

\begin{frame}[fragile]

\frametitle{Request handler: the handler itself}

\begin{minted}{lua}
-- Handler code is checked for access to illegal globals.
-- Legal globals are aliased to locals at the top.
-- Necessary require() calls are added automatically.

local handler = function(param)
  return
  {
    reversed = param.text:reverse();
  }
end

HANDLERS["/reverse.xml"] = handler;
HANDLERS["/reverse.json"] = handler;
\end{minted}

\end{frame}

%% -------------------------------------------------------------------------- %%

\begin{frame}[fragile]

\frametitle{Documentation}
\large{\textbf{/reverse.\{xml,json\}: String reverser}}

Takes a string and reverses it.

\textbf{IN}
\begin{semiverbatim}
  ?text=STRING
\end{semiverbatim}

\textbf{OUT}

\textit{XML:}

\begin{minted}{xml}
<result reversed="STRING" />
\end{minted}

\textit{JSON:}

\begin{minted}{javascript}
{ "result": { "reversed": "STRING" } }
\end{minted}

\end{frame}

%% -------------------------------------------------------------------------- %%

\begin{frame}[fragile]

\frametitle{Smoke tests}

\begin{minted}{lua}
test:case "/reverse.xml:smoke.ok" (function()
  local reply = assert(http.GET(
      TEST_HOST .. "/reverse.xml?text=Foo")
    ))
  assert(type(reply.result) == "table")
  assert(type(reply.result.reversed) == "string")
end)
\end{minted}

\end{frame}

%% -------------------------------------------------------------------------- %%

\begin{frame}

\frametitle{Too complicated for ad-hoc solution!}

TODO: Spaghetti cat image goes here

\end{frame}

%% -------------------------------------------------------------------------- %%

\section { The "proper" solution }

%% -------------------------------------------------------------------------- %%

\begin{frame}

\frametitle{The "proper" solution?}

\begin{itemize}
\item Should be easy to add a new target.
\item Should have nicer error reporting.
\item Should be reusable.
\end{itemize}

\end{frame}

%% -------------------------------------------------------------------------- %%

\begin{frame}

\frametitle{The flow}

\begin{itemize}
\item Load data
\item Validate correctness
\item Generate output
\end{itemize}

\end{frame}

%% -------------------------------------------------------------------------- %%

\begin{frame}[fragile]

\frametitle{Let's recap how our data looks like}

\begin{minted}{lua}
api:url "/reverse"
{
  doc:description [[String reverser]]
  [[
    Takes a string and reverses it.
  ]]
  api:input { data:string "text" };
  api:output
  {
    data:node "result" { data:string "reversed" };
  };
  handler = function(param)
    return { reversed = param.text:reverse() }
  end;
}
\end{minted}

\end{frame}

%% -------------------------------------------------------------------------- %%

\begin{frame}[fragile]

\frametitle{Surprise! It's a tree!}

\begin{minted}{lua}
{ id = "api:url", name = "/reverse";
  { id = "doc:description", name = "String reverser";
    text = "Takes a string and reverses it.";
  };
  { id = "api:input";
    { id = "data:string", name = "text" };
  };
  { id = "api:output";
    { id = "data:node", name = "result";
      { id = "data:string", name = "reversed" };
    };
    handler = function(param)
      return { reversed = param.text:reverse() }
    end;
  };
}
\end{minted}

\end{frame}

%% -------------------------------------------------------------------------- %%

\begin{frame}[fragile]

\frametitle{We need a loader, that does this:}

\begin{columns}

\begin{column}{0.5\textwidth}
\begin{minted}{lua}
namespace:method "title"
\uncover<2->{
{
  data = "here";
}}
\end{minted}
\end{column}

\begin{column}{0.05\textwidth}
$\Rightarrow$
\end{column}

\begin{column}{0.5\textwidth}
\begin{minted}{lua}
{
  id = "namespace:method";
  title = "title";
\uncover<2->{data = "here";}
}
\end{minted}
\end{column}

\end{columns}

\end{frame}

%% -------------------------------------------------------------------------- %%



%% -------------------------------------------------------------------------- %%


%% -------------------------------------------------------------------------- %%



\end{document}

%% -------------------------------------------------------------------------- %%
