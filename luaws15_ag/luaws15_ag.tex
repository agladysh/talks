\documentclass[handout]{beamer}

\usepackage{minted}
\usepackage[utf8]{inputenc}
\usepackage{graphicx}
\usepackage[T1]{fontenc}

\mode<presentation>{
  \usetheme{default}
}

\setbeamertemplate{navigation symbols}{}
\setbeamertemplate{footline}[frame number]

\newcommand{\comment}[1]{}

%% -------------------------------------------------------------------------- %%

\title{Ad-hoc Big-Data Analysis with Lua}
\subtitle{And LuaJIT}
\author{\includegraphics[height=.4\textheight]{logo}}
\institute{Alexander Gladysh <ag@logiceditor.com>\newline@agladysh}
\date{Lua Workshop 2015\\Stockholm}

%% -------------------------------------------------------------------------- %%

\begin{document}

\maketitle

%% -------------------------------------------------------------------------- %%

\begin{frame}{Outline}

\tableofcontents

\end{frame}

%% -------------------------------------------------------------------------- %%

\section{Introduction}

%% -------------------------------------------------------------------------- %%

\begin{frame}{Alexander Gladysh}

\begin{itemize}
\item CTO, co-owner at LogicEditor
\item In löve with Lua since 2005
\end{itemize}

\end{frame}

%% -------------------------------------------------------------------------- %%

\section{The Problem}

%% -------------------------------------------------------------------------- %%

\begin{frame}{The Problem}

\begin{itemize}
\item You have a dataset to analyze,
\item which is too large for "small-data" tools,
\item and have no resources to setup and maintain (or pay for)
      the Hadoop, Google Big Query etc.
\item but you have some processing power available.
\end{itemize}

\end{frame}

%% -------------------------------------------------------------------------- %%

\section{A Solution}

%% -------------------------------------------------------------------------- %%

\begin{frame}{Goal}

\begin{itemize}
\item Pre-process the data so it can be handled by R or Excel or your favorite
analytics tool (or Lua!).
\item If the data is dynamic, then \textit{learn to} pre-process it
      and build a data processing pipeline.
\end{itemize}

\end{frame}

%% -------------------------------------------------------------------------- %%

\begin{frame}{An Approach}

\begin{itemize}
\item Use Lua!
\item And (semi-)standard tools, available on Linux.
\item Go minimalistic while exploring, avoid frameworks,
\item Then move on to an industrial solution that fits your newly understood
      requirements,
\item Or roll your own ecosystem! ;-)
\end{itemize}

\end{frame}

%% -------------------------------------------------------------------------- %%

\section{Assumptions}

%% -------------------------------------------------------------------------- %%

\begin{frame}
\huge Assumptions
\end{frame}

%% -------------------------------------------------------------------------- %%

\begin{frame}{Data Format}

\begin{itemize}
\item Plain text
\item Column-based (csv-like), optionally with free-form data in the end
\item Typical example: web-server log files
\end{itemize}

\end{frame}

%% -------------------------------------------------------------------------- %%

\begin{frame}[fragile]{Data Format Example: Raw Data}

\begin{minted}{text}
2015/10/15 16:35:30 [info] 14171#0: *901195
[lua] index:14: 95c1c06e626b47dfc705f8ee6695091a
109.74.197.145 *.example.com
GET 123456.gif?q=0&step=0&ref= HTTP/1.1 example.com
\end{minted}
\textit{NB: This is a single, tab-separated line from a time-sorted file.}

\end{frame}

%% -------------------------------------------------------------------------- %%

\begin{frame}[fragile]{Data Format Example: Intermediate Data}

\begin{minted}{text}
alpha.example.com	5
beta.example.com	7
gamma.example.com	1
\end{minted}

\textit{NB: These are several tab-separated lines from a key-sorted file.}

\end{frame}

%% -------------------------------------------------------------------------- %%

\begin{frame}{Hardware}

\begin{itemize}
\item As usual, more is better: Cores, cache, memory speed and size,
      HDD speeds, networking speeds...
\item But even a modest VM (or several) can be helpful.
\item Your fancy gaming laptop is good too ;-)
\end{itemize}

\end{frame}

%% -------------------------------------------------------------------------- %%

\begin{frame}{OS}

\Large Linux (Ubuntu) Server.

\textit{This approach will, of course, work for other setups.}

\end{frame}

%% -------------------------------------------------------------------------- %%

\begin{frame}{Filesystem}

\begin{itemize}
\item Ideally, have data copies on each processing node, using identical
      layouts.
\item Fast network should work too.
\end{itemize}

\end{frame}

%% -------------------------------------------------------------------------- %%

\section{Examples}

%% -------------------------------------------------------------------------- %%

\begin{frame}
\huge Examples
\end{frame}

%% -------------------------------------------------------------------------- %%

\begin{frame}[fragile]{Bash Script Example}

\begin{minted}{bash}
time pv /path/to/uid-time-url-post.gz \
| pigz -cdp 4 \
| cut -d$'\t' -f 1,3 \
| parallel --gnu --progress -P 10 --pipe --block=16M \
  $(cat <<"EOF"
    luajit ~me/url-to-normalized-domain.lua
EOF
  ) \
| LC_ALL=C sort -u -t$'\t' -k2 --parallel 6 -S20% \
| luajit ~me/reduce-value-counter.lua \
| LC_ALL=C sort -t$'\t' -nrk2 --parallel 6 -S20% \
| pigz -cp4 >/path/to/domain-uniqs_count-merged.gz
\end{minted}

\end{frame}

%% -------------------------------------------------------------------------- %%

\begin{frame}[fragile]{Lua Script Example: url-to-normalized-domain.lua}

\begin{minted}{bash}
for l in io.lines() do
  local key, value = l:match("^([^\t]+)\t(.*)")
  if value then
    value = url_to_normalized_domain(value)
  end
  if key and value then
    io.write(key, "\t", value, "\n")
  end
end
\end{minted}

\end{frame}

%% -------------------------------------------------------------------------- %%

\begin{frame}[fragile]{Lua Script Example: reduce-value-counter.lua 1/3}

\begin{minted}{lua}
-- Assumes input sorted by VALUE
-- a  foo --> foo  3
-- a  foo     bar  2
-- b  foo     quo  1
-- a  bar
-- c  bar
-- d  quo
\end{minted}

\end{frame}

%% -------------------------------------------------------------------------- %%

\begin{frame}[fragile]{Lua Script Example: reduce-value-counter.lua 2/3}

\begin{minted}{lua}
local last_key = nil, accum = 0

local flush = function(key)
  if last_key then
    io.write(last_key, "\t", accum, "\n")
  end
  accum = 0
  last_key = key -- may be nil
end
\end{minted}

\end{frame}

%% -------------------------------------------------------------------------- %%

\begin{frame}[fragile]{Lua Script Example: reduce-value-counter.lua 3/3}

\begin{minted}{lua}
for l in io.lines() do
  -- Note reverse order!
  local value, key = l:match("^(.-)\t(.*)$")
  assert(key and value)

  if key ~= last_key then
    flush(key)
    collectgarbage("step")
  end

  accum = accum + 1
end

flush()
\end{minted}

\end{frame}

%% -------------------------------------------------------------------------- %%

\begin{frame}{Tying It All Together}

Basically:
\begin{itemize}
\item You work with sorted data,
\item mapping and reducing it line-by-line,
\item in parallel where at all possible,
\item while trying to use as much of available hardware resources as practical,
\item and without running out of memory.
\end{itemize}

\end{frame}

%% -------------------------------------------------------------------------- %%

\section{The Tools}

%% -------------------------------------------------------------------------- %%

\begin{frame}
\huge The Tools
\end{frame}

%% -------------------------------------------------------------------------- %%

\begin{frame}{The Tools}

\begin{itemize}
\item parallel
\item sort, uniq, grep
\item cut, join, comm
\item pv
\item compression utilities
\item LuaJIT
\end{itemize}

\end{frame}

%% -------------------------------------------------------------------------- %%

\begin{frame}{LuaJIT?}

Up to a point:
\begin{itemize}
\item 2.1 helps to speed things up,
\item FFI bogs down development speed.
\item Go plain Lua first (run it with LuaJIT),
\item then roll your own ecosystem as needed ;-)
\end{itemize}

\end{frame}

%% -------------------------------------------------------------------------- %%

\begin{frame}{Parallel}

\begin{itemize}
\item xargs for parallel computation
\item can run your jobs in parallel on a single machine
\item or on a "cluster"
\end{itemize}

\end{frame}

%% -------------------------------------------------------------------------- %%

\begin{frame}{Compression}

\begin{itemize}
\item gzip: default, bad
\item lz4: fast, large files
\item pigz: fast, parallelizable
\item xz: good compression, slow
\item ...and many more,
\item be on lookout for new formats!
\end{itemize}

\end{frame}

%% -------------------------------------------------------------------------- %%

\begin{frame}[fragile]{GNU sort Tricks}

\begin{minted}{bash}
LC_ALL=C \
sort -t$'\t' --parallel 4 -S60% \
-k3,3nr -k2,2  -k1,1nr
\end{minted}

\begin{itemize}
\item Disable locale.
\item Specify delimiter.
\item Note that parallel x4 with 60\% memory
      will consume 0.6 * log(4) = 120\% of memory.
\item When doing multi-key sort, specify parameters after key number.
\end{itemize}

\end{frame}

%% -------------------------------------------------------------------------- %%

\begin{frame}{grep}

http://stackoverflow.com/questions/9066609/fastest-possible-grep

\end{frame}

%% -------------------------------------------------------------------------- %%

\section{Notes}

%% -------------------------------------------------------------------------- %%

\begin{frame}
\huge Notes and Remarks
\end{frame}

%% -------------------------------------------------------------------------- %%

\begin{frame}{Why Lua?}

Perl, AWK are traditional alternatives to Lua,
but, if you're not very disciplined and experienced,
they are much less maintainable.

\end{frame}

%% -------------------------------------------------------------------------- %%

\begin{frame}{Start Small!}

\begin{itemize}
\item Always run your scripts on small representative excerpts
      from your datasets, not only while developing them locally,
      but on actual data-processing nodes too.
\item Saves time and helps you learn the bottlenecks.
\item Sometimes large run still blows in your face though:
\item Monitor resource utilization at run-time.
\end{itemize}

\end{frame}

%% -------------------------------------------------------------------------- %%

\begin{frame}{Discipline!}

\begin{itemize}
\item Many moving parts, large turn-around times, hard to keep tabs.
\item Keep journal: Write down what you run and what time it took.
\item Store actual versions of your scripts in a source control system.
\item Don't forget to sanity-check the results you get!
\end{itemize}

\end{frame}

%% -------------------------------------------------------------------------- %%

\section{Questions?}

%% -------------------------------------------------------------------------- %%

\begin{frame}{Questions?}

Alexander Gladysh,
ag@logiceditor.com

\end{frame}

%% -------------------------------------------------------------------------- %%

\end{document}

%% -------------------------------------------------------------------------- %%
