\documentclass[handout]{beamer}

\usepackage{minted}
\usepackage[utf8]{inputenc}
\usepackage{graphicx}
\usepackage[T1,T2A]{fontenc}
\usepackage[american,russian]{babel}

%% -------------------------------------------------------------------------- %%

\mode<presentation>{
  \usetheme{default}
}

\setbeamertemplate{navigation symbols}{}
\setbeamertemplate{footline}[frame number]

\newcommand{\comment}[1]{}

%% -------------------------------------------------------------------------- %%

\title{\includegraphics[height=.15\textheight]{logo}}
\author{Пользовательская автоматизация\newlineпрофессиональных веб-приложений\newlineна Lua}
\institute{Александр Гладыш\newline@agladysh}
\date{Lua in Moscow\\Сентябрь 2016}

%% -------------------------------------------------------------------------- %%

\begin{document}

\maketitle

%% -------------------------------------------------------------------------- %%

\begin{frame}{План}

\tableofcontents

\end{frame}

%% -------------------------------------------------------------------------- %%

\section{Задача}

%% -------------------------------------------------------------------------- %%

\begin{frame}{Что такое профессиональные приложения?}

\begin{itemize}
\item CAD'ы
\item Приложения для управления бизнес-логикой "больших" систем
      (например, в авиации)
\item Бэкофисы сложных проектов
\item и так далее
\end{itemize}

\end{frame}

%% -------------------------------------------------------------------------- %%

\begin{frame}{Кто пользователи?}

\begin{itemize}
\item Профессионалы, глубоко владеющие предметной областью
\item Не IT-шники
\end{itemize}

\end{frame}

%% -------------------------------------------------------------------------- %%

\begin{frame}{Конкретика}

\begin{itemize}
\item ПО для гражданской авиации
\item Глубочайшая предметная область
\end{itemize}

\end{frame}

%% -------------------------------------------------------------------------- %%

\begin{frame}{Архитектура}

\begin{itemize}
\item TODO: Диаграмма. Клиент -- вершина айсберга.
\end{itemize}

\end{frame}

%% -------------------------------------------------------------------------- %%

\begin{frame}{Задачи}

\begin{itemize}
\item Автоматизация редких но сложных операций пользователя-эксперта
\item В дальнейшем --- более высокоуровневое API для более частых команд
\end{itemize}

\end{frame}

%% -------------------------------------------------------------------------- %%

\section{Решение}

%% -------------------------------------------------------------------------- %%

\begin{frame}{Почему макросы на клиенте?}

\begin{itemize}
\item TODO
\end{itemize}

\end{frame}

%% -------------------------------------------------------------------------- %%

\begin{frame}{Почему Lua?}

\begin{itemize}
\item TODO
\item Изоляция интерфейса
\item Лёгкость в освоении
\end{itemize}

\end{frame}

%% -------------------------------------------------------------------------- %%

\begin{frame}{Lua в браузере: варианты}

\begin{itemize}
\item TODO
\item Изоляция интерфейса
\item Лёгкость в освоении
\end{itemize}

\end{frame}

%% -------------------------------------------------------------------------- %%

\begin{frame}{Решение: стек}

\begin{itemize}
\item TODO
\end{itemize}

\end{frame}

%% -------------------------------------------------------------------------- %%

\begin{frame}{Решение: внешний вид}

\begin{itemize}
\item TODO
\end{itemize}

\end{frame}

%% -------------------------------------------------------------------------- %%

\begin{frame}{Что дальше?}

\begin{itemize}
\item TODO
\end{itemize}

\end{frame}

%% -------------------------------------------------------------------------- %%

\section{Вопросы?}

%% -------------------------------------------------------------------------- %%

\begin{frame}{Вопросы?}

Alexander Gladysh\newline
@agladysh

\end{frame}

%% -------------------------------------------------------------------------- %%

\end{document}

%% -------------------------------------------------------------------------- %%
