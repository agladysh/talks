% Build with xetex.
\documentclass[aspectratio=169,handout,bigger]{beamer}

%% -------------------------------------------------------------------------- %%

\usepackage{xunicode}
\usepackage{xltxtra}
\usepackage{graphicx}

\usepackage{color}

\usepackage{setspace}
\usepackage{ragged2e}

\usepackage[normalem]{ulem}

\newcommand{\soutt}[1]{%
    \renewcommand{\ULthickness}{1pt}%
       \sout{#1}%
    \renewcommand{\ULthickness}{.4pt}% Resetting to ulem default
}

\usepackage{minted}

\usepackage{tikz}
\usetikzlibrary{shapes,arrows,chains,calc,fit,matrix}

%% -------------------------------------------------------------------------- %%

\usepackage{polyglossia}
\setmainlanguage{russian}
\setotherlanguage{english}

% NB: To get MS fonts for OS X:
%
% 1) Install MS Open XML Converter.
% 2) Install OpenXML_all_fonts.pkg from inside of Converter's mpkg.
%
% http://www.labnol.org/software/tutorials/\
% free-download-calibri-font-on-mac/3684/

\setmainfont{Calibri}
\setsansfont{Calibri}
\setmonofont{Consolas}

%% -------------------------------------------------------------------------- %%

\mode<presentation>{
  \usetheme{default}
}

\useinnertheme{circles}

\setbeamertemplate{navigation symbols}{}
\setbeamertemplate{section in toc}[sections numbered]

\definecolor{chart11}{RGB}{0, 0, 0}
\setbeamercolor{title}{fg=chart11}
\setbeamercolor{author}{fg=chart11}
\setbeamercolor{frametitle}{fg=chart11}
\setbeamercolor{itemize item}{fg=chart11}
\setbeamercolor{itemize subitem}{fg=chart11}
\setbeamercolor{itemize subsubitem}{fg=chart11}
\setbeamercolor{section in toc}{fg=chart11}

\setbeamerfont{title}{series=\bfseries,parent=structure}

\defbeamertemplate*{title page}{customized}[1][]
{
  \vbox{}
  \vfill
  \begingroup
    \centering
    \vspace*{-2.5cm}
    \begin{beamercolorbox}[sep=8pt,left,#1]{title}
      \usebeamerfont{title}\inserttitle\par%
    \end{beamercolorbox}%
    \par
    \begin{beamercolorbox}[sep=8pt,left,#1]{author}
      \usebeamerfont{author}\insertauthor
    \end{beamercolorbox}
    {\usebeamercolor[fg]{titlegraphic}\inserttitlegraphic\par}
  \endgroup
  \vfill
}

\defbeamertemplate{footline}{left page number}
{%
  \hspace*{1em}
  \usebeamercolor[fg]{page number in head/foot}%
  \usebeamerfont{page number in head/foot}%
  \insertpagenumber\,/\,\insertpresentationendpage%
  \vskip1em%
}
\setbeamertemplate{footline}[left page number]

%% ========================================================================== %%

\title{Быстрое прототипирование \\
бэкенда игры с геолокацией \\
на OpenResty, Redis и Docker}
\author{Александр Гладыш\\CTO, LogicEditor}
\date{}

%% ========================================================================== %%

\begin{document}

\usebackgroundtemplate{
  \includegraphics[width=\paperwidth,height=\paperheight]{back_title}
}

% Using [plain] to avoid frame number on the title page.
\begin{frame}[plain]
 \titlepage
\end{frame}

%% ========================================================================== %%

\usebackgroundtemplate{
  \includegraphics[width=\paperwidth,height=\paperheight]{back}
}

\begin{frame}{Содержание}

\tableofcontents

\end{frame}

%% ========================================================================== %%

\section*{}

\begin{frame}{Обо мне}

\begin{itemize}
\item В разработке ПО с 2002-го,
\item большую часть этого времени --- в геймдеве (разработка, проектирование, управление),
\item вне геймдева --- нагруженные интернет-решения, enterprise ПО и др.
\item Организатор \href{http://meetup.com/Lua-in-Moscow}{meetup.com/Lua-in-Moscow}
\end{itemize}

\end{frame}

%% ========================================================================== %%

\section{Кейс}

%% ========================================================================== %%

\begin{frame}{Мобильные игры с геолокацией}
  \centering{\includegraphics[height=.75\textheight]{pokemongo}}
\end{frame}

%% -------------------------------------------------------------------------- %%

\begin{frame}{Цели прототипирования в общем}
  \begin{itemize}
    \item Проверить ряд подходов к построению игрового процесса,
    \item выработать новые идеи,
    \item найти, в чём фан, а в чём --- нет.
  \end{itemize}
\end{frame}

%% -------------------------------------------------------------------------- %%

\begin{frame}{Цели технологической части прототипирования}
  \begin{itemize}
    \item С минимальными затратами получить код,
    \item дающий возможность быстро итерироваться по вариантам геймплея.
    \item Прояснить на практике технические ограничения жанра.
  \end{itemize}
\end{frame}

%% -------------------------------------------------------------------------- %%

\begin{frame}{Результаты}
  За два календарных месяца (менее 100 человеко-часов)
  разработан прототип серверной части игры с геолокацией
  и рудиментарный клиент для неё.
  \vspace*{1em}\par
  Ведётся быстрое итерирование по вариантам построения игрового процесса
  и дальнейшая разработка технологической части проекта.
\end{frame}

%% -------------------------------------------------------------------------- %%

\begin{frame}{О чём этот доклад?}
  \begin{itemize}
    \item Доклад --- технологической части проекта,
    \item не о геймдизайне
    \item и не о монетизации.
  \end{itemize}
\end{frame}

%% -------------------------------------------------------------------------- %%

\begin{frame}{Зачем этот доклад?}
  Делать игры с геолокацией сейчас проще чем когда-либо.
  \vspace*{1em}\par
  Я покажу как начать.
\end{frame}

%% ========================================================================== %%

\section{Задача}

%% ========================================================================== %%

\begin{frame}{Задача}
  \begin{itemize}
    \item Максимально быстро написать сервер для быстрого прототипирования.
    \item Параллельно с сервером реализовать минимальный клиент.
    \item Проверять гипотезы уже во время написания, если возможно.
    \item Выявить основные технические ограничения на проект.
  \end{itemize}
\end{frame}

%% ========================================================================== %%

\section{Философия}

%% ========================================================================== %%

\begin{frame}{Стадии разработки}
  \begin{itemize}
    \item Препродакшен
    \item Продакшен
    \item Поддержка
  \end{itemize}
\end{frame}

%% -------------------------------------------------------------------------- %%

\begin{frame}{Стадии разработки: Препродакшен}
  \begin{itemize}
    \item Препродакшен
    \begin{itemize}
      \item Поиск геймплея, эскизы, выяснение ограничений.
      \item Более глубокая проработка удачных вариантов геймплея.
      \item Подготовка к продакшену выбранного варианта.
    \end{itemize}
  \end{itemize}
\end{frame}

%% -------------------------------------------------------------------------- %%

\begin{frame}{Продакшен: приоритеты разработки}
  \begin{itemize}
    \item\huge{Восприятие проекта}
    \item\LARGE{Устойчивость ко взлому}
    \item\Large{Масштабируемость}
    \item\Large{Производительность}
    \item\large{Стабильность}
    \item\normalsize{Гибкость}
    \item\normalsize{Скорость итерирования}
    \item\normalsize{Простота}
  \end{itemize}
\end{frame}

%% -------------------------------------------------------------------------- %%

\begin{frame}{Препродакшен: время на вес золота}
  \begin{itemize}
    \item\huge{Скорость итерирования}
    \item\large{Гибкость}
    \item\large{Простота}
    \item\normalsize{Восприятие проекта}
    \item\soutt{\small{Стабильность}}
    \item\soutt{\small{Производительность}}
    \item\soutt{\small{Масштабируемость}}
    \item\soutt{\small{Устойчивость ко взлому}}
  \end{itemize}
\end{frame}

%% -------------------------------------------------------------------------- %%

\begin{frame}{Фокус на скорость и лёгкость разработки}
  \begin{itemize}
    \item Механизмы, а не решения.
    \item Лучше быстро чем правильно.
    \item Много коротких итераций.
  \end{itemize}
\end{frame}

%% -------------------------------------------------------------------------- %%

\begin{frame}{Лучше быстро чем правильно}
  \begin{itemize}
    \item Чем меньше кода тем лучше.
    \item Плохой код лучше сложного.
    \item Хаки в механизмах допустимы, хаки в решениях --- нет.
    \item Рефакторить нужно только то, что болит.
  \end{itemize}
\end{frame}

%% ========================================================================== %%

\section{План разработки}

%% ========================================================================== %%

\begin{frame}{Основные этапы разработки}
  \begin{itemize}
    \item Выбор геймплея пилотного прототипа.
    \item Выбор технологического стека.
    \item Реализация минимального пилотного прототипа.
    \item Итерирование с гейм-дизайнерами.
  \end{itemize}
\end{frame}

%% -------------------------------------------------------------------------- %%

\begin{frame}{Выбор геймплея пилотного прототипа: Задачи}
  \begin{itemize}
    \item Максимально простой
    \item но играбельный
    \item повод реализовать все базовые игровые сущности и механизмы прототипа.
  \end{itemize}
\end{frame}

%% -------------------------------------------------------------------------- %%

\begin{frame}{Геймплей первого прототипа}
  \begin{itemize}
    \item Игрок, перемещаясь по карте, ищет расставленных на ней мобов;
    \item найдя моба может попробовать его поймать с заданной вероятностью успеха;
    \item успешная поимка моба увеличивает счётчик в характеристиках игрока;
    \item пойманный моб исчезает с карты;
    \item через некоторое время моб респавнится в том же месте;
    \item администраторы могут добавлять новых мобов на карту.
  \end{itemize}
\end{frame}

%% -------------------------------------------------------------------------- %%

\begin{frame}{Критерии выбора технологического стека}
  \begin{itemize}
    \item Что-то знакомое, на чём можно написать быстро,
    \item или что-то интересное и зажигающее.
    \item Главное не забыть вовремя выбросить весь код.
  \end{itemize}
\end{frame}

%% -------------------------------------------------------------------------- %%

\begin{frame}{Технологический стек}
  \begin{itemize}
    \item Сервер:
    \begin{itemize}
      \item Redis,
      \item OpenResty,
      \item Docker.
    \end{itemize}
    \item Клиент:
    \begin{itemize}
      \item Одностраничное веб-приложение в браузере,
      \item HTML5.
    \end{itemize}
  \end{itemize}
\end{frame}

%% -------------------------------------------------------------------------- %%

\begin{frame}{Почему не что-то готовое?}
  Not Invented Here Syndrome?
  \vspace*{1em}\par
  И да и нет.
\end{frame}

%% -------------------------------------------------------------------------- %%

\begin{frame}{Redis}
  \begin{itemize}
    \item Надёжное, хорошо зарекомендовавшее себя решение.
    \item Работа с координатами из коробки:
    \begin{itemize}
      \item \texttt{GEOADD key longitude latitude member}
      \item \texttt{GEORADIUS key longitude latitude radius m}
    \end{itemize}
    \item Достаточно удобный набор примитивов для хранения игровых объектов.
    \item Хранимые процедуры на Lua.
  \end{itemize}
\end{frame}

%% -------------------------------------------------------------------------- %%

\begin{frame}{OpenResty}
  \begin{itemize}
    \item Дистрибутив nginx с поддержкой Lua, Redis и многим другим из коробки.
    \item Очень быстро работает, достаточно дружелюбен, хорошо поддерживается.
    \item Пригоден как для быстрого прототипирования, так и для продакшена.
  \end{itemize}
\end{frame}

%% -------------------------------------------------------------------------- %%

\begin{frame}{Docker}
  \begin{itemize}
    \item Воспроизводимая кроссплатформенная среда для разработки.
    \item Хорошо снимает боль по настройке окружения разработчика.
    \item Окружение разработчика можно быстро превратить в прототип серверного окружения.
    \item Требует обновления до достаточно свежей версии.
  \end{itemize}
\end{frame}

%% -------------------------------------------------------------------------- %%

\begin{frame}{Браузер, HTML5}
  \begin{itemize}
    \item На начальном этапе сервер важнее.
    \item Бои в augmented reality и прочие рюшечки делать не нужно,
          можно представлять в голове.
    \item На HTML5 можно быстро написать дёшевый и сердитый клиент.
    \item Есть ограниченный (но достаточный) доступ к данным геолокации.
  \end{itemize}
\end{frame}

%% ========================================================================== %%

\section{Docker}

%% ========================================================================== %%

\begin{frame}{Docker: как установить на Ubuntu}
  \begin{itemize}
    \item В Ubuntu, традиционно, старая версия.
    \item За \texttt{wget | sh} больно бьём по рукам.
    \item \texttt{docker} и \texttt{docker-machine}
          устанавливаем из apt-репозитория докера.
    \item \texttt{docker-compose} устанавливаем через \texttt{pip install}.
    \item Про установку на других платформах читаем официальную документацию.
  \end{itemize}
\end{frame}

%% -------------------------------------------------------------------------- %%

\begin{frame}{Docker на машине разработчика}

\begin{center}
\begin{tikzpicture}
  every join/.style={line},    % Default linetype for connecting boxes
  scale=1, transform shape
]

\tikzset{
  basebase/.style={
    align=center, minimum height=2em, minimum width=9em, color=chart11,
    inner sep=.5em, node distance=3em and 11em
  },
  rect/.style={
    basebase, draw, on grid
  },
  coord/.style={coordinate},
  line/.style={->, draw, chart11},
}

\node[rect, rounded corners]                     (client)    {Клиент};
\node[rect, rounded corners, below of=client]    (docker)    {localhost:8080};
\node[rect, below of=docker]    (openresty) {OpenResty};
\node[rect, below of=openresty] (redis)     {Redis};

\draw [line] (client)    -- (docker);
\draw [line] (docker)    -- (openresty);
\draw [line] (openresty) -- (redis);

\end{tikzpicture}
\end{center}

\end{frame}

%% -------------------------------------------------------------------------- %%

\begin{frame}[fragile]{docker-compose.yml для разработки: Redis}
\begin{minted}{yaml}
version: "2"
services:
  redis:
    image: redis
    volumes:
      - ./redis:/data
    command: redis-server --appendonly yes
  openresty: <...>
\end{minted}
\end{frame}

%% -------------------------------------------------------------------------- %%

\begin{frame}[fragile]{OpenResty: интересные места nginx.conf (в сокращении)}
\begin{minted}{nginx}
error_log logs/error.log notice;
http {
  include resolvers.conf;
  lua_package_path "$prefix/lualib/?.lua;;";
  lua_code_cache off; # TODO: Enable on production!
  server {
    listen 8080;
    include mime.types;
    default_type application/json;
    location / { index index.html; root static/; }
    location = /api/v1/ { content_by_lua_file 'api/index.lua'; }
  }
}
\end{minted}
\end{frame}

%% -------------------------------------------------------------------------- %%

\begin{frame}[fragile]{OpenResty: Dockerfile}
\begin{minted}{docker}
FROM openresty/openresty
COPY bin/entrypoint.sh /usr/local/bin/openresty-entrypoint.sh
COPY nginx/conf /usr/local/openresty/nginx/conf
COPY nginx/lualib /usr/loca/openresty/nginx/lualib
COPY nginx/lua /usr/loca/openresty/nginx/lua
COPY nginx/static /usr/loca/openresty/nginx/static
ENTRYPOINT /usr/local/bin/openresty-entrypoint.sh
\end{minted}
\end{frame}

%% -------------------------------------------------------------------------- %%

\begin{frame}[fragile]{OpenResty: entrypoint.sh}
\begin{minted}{bash}
#!/bin/sh
grep nameserver /etc/resolv.conf \
  | awk '{print  "resolver " $2 ";"}' \
  > /usr/local/openresty/nginx/conf/resolvers.conf
/usr/local/openresty/bin/openresty -g 'daemon off;' "$@"
\end{minted}
\end{frame}

%% -------------------------------------------------------------------------- %%

\begin{frame}[fragile]{docker-compose.yml для разработки: OpenResty}
\begin{minted}{yaml}
  <...>
  openresty:
    build: .
    volumes:
      - ./nginx/lualib:/usr/local/openresty/nginx/lualib:ro
      - ./nginx/api:/usr/local/openresty/nginx/api:ro
      - ./nginx/static:/usr/local/openresty/nginx/static:ro
      links:
        - redis
\end{minted}
\end{frame}

%% -------------------------------------------------------------------------- %%

\begin{frame}{код: проектирование механизмов для первого прототипа, задачи}
  \begin{itemize}
    \item TODO
  \end{itemize}
\end{frame}

%% -------------------------------------------------------------------------- %%

\begin{frame}{игровой объект: обзор}
  \begin{itemize}
    \item TODO
  \end{itemize}
\end{frame}

%% -------------------------------------------------------------------------- %%

\begin{frame}{игровой объект: характеристики}
  \begin{itemize}
    \item TODO
  \end{itemize}
\end{frame}

%% -------------------------------------------------------------------------- %%

\begin{frame}{игровой объект: прототипы}
  \begin{itemize}
    \item TODO
  \end{itemize}
\end{frame}

%% -------------------------------------------------------------------------- %%

\begin{frame}{игровой объект: действия}
  \begin{itemize}
    \item TODO
  \end{itemize}
\end{frame}

%% -------------------------------------------------------------------------- %%

\begin{frame}{игровой объект: права на действия}
  \begin{itemize}
    \item TODO
  \end{itemize}
\end{frame}

%% -------------------------------------------------------------------------- %%

\begin{frame}{игровой объект: координаты}
  \begin{itemize}
    \item TODO
  \end{itemize}
\end{frame}

%% -------------------------------------------------------------------------- %%

\begin{frame}{игровой объект: как это ложится на базу, плевать на производительность, плевать на атомарность, плевать на читеров}
  \begin{itemize}
    \item TODO
  \end{itemize}
\end{frame}

%% -------------------------------------------------------------------------- %%

\begin{frame}{игровой объект: почему так?}
  \begin{itemize}
    \item TODO
  \end{itemize}
\end{frame}

%% -------------------------------------------------------------------------- %%

\begin{frame}{апи: статус}
  \begin{itemize}
    \item TODO
  \end{itemize}
\end{frame}

%% -------------------------------------------------------------------------- %%

\begin{frame}{апи: действия}
  \begin{itemize}
    \item TODO
  \end{itemize}
\end{frame}

%% -------------------------------------------------------------------------- %%

\begin{frame}{апи: отложенные действия}
  \begin{itemize}
    \item TODO
  \end{itemize}
\end{frame}

%% -------------------------------------------------------------------------- %%

\begin{frame}{апи: нет админки (пока), есть внутриигровые админские действия}
  \begin{itemize}
    \item TODO
  \end{itemize}
\end{frame}

%% -------------------------------------------------------------------------- %%

\begin{frame}{апи: сброс и патч базы; проще, ещё проще!}
  \begin{itemize}
    \item TODO
  \end{itemize}
\end{frame}

%% -------------------------------------------------------------------------- %%

\begin{frame}{клиент: первый клиент это curl, демонстрация curl}
  \begin{itemize}
    \item TODO
  \end{itemize}
\end{frame}

%% -------------------------------------------------------------------------- %%

\begin{frame}{клиент: демонстрация html5}
  \begin{itemize}
    \item TODO
  \end{itemize}
\end{frame}

%% -------------------------------------------------------------------------- %%

\begin{frame}{клиент: html5 геолокация, https-only (кроме localhost)}
  \begin{itemize}
    \item TODO
  \end{itemize}
\end{frame}

%% -------------------------------------------------------------------------- %%

\begin{frame}{клиент: как работает версия на js}
  \begin{itemize}
    \item TODO
  \end{itemize}
\end{frame}

%% -------------------------------------------------------------------------- %%

\begin{frame}{клиент, трюки: имена объектов}
  \begin{itemize}
    \item TODO
  \end{itemize}
\end{frame}

%% -------------------------------------------------------------------------- %%

\begin{frame}{клиент, трюки: гуглекарта}
  \begin{itemize}
    \item TODO
  \end{itemize}
\end{frame}

%% -------------------------------------------------------------------------- %%

\begin{frame}{клиент, трюки: асинхронная перерисовка}
  \begin{itemize}
    \item TODO
  \end{itemize}
\end{frame}

%% -------------------------------------------------------------------------- %%

\begin{frame}{три вектора развития: геймплей, технологии, фичи; приоритеты; не закопаться!}
  \begin{itemize}
    \item TODO
  \end{itemize}
\end{frame}

%% -------------------------------------------------------------------------- %%

\begin{frame}{как работать с фидбеком от геймдизайнеров: механизмы, не хаки}
  \begin{itemize}
    \item TODO
  \end{itemize}
\end{frame}

%% -------------------------------------------------------------------------- %%

\begin{frame}{как работать с фидбеком от геймдизайнеров: приоритеты: баги, новые механизмы, далее по убыванию боли}
  \begin{itemize}
    \item TODO
  \end{itemize}
\end{frame}

%% -------------------------------------------------------------------------- %%

\begin{frame}{как работать с фидбеком от геймдизайнеров: быстрые итерации, садиться рядом и кодить}
  \begin{itemize}
    \item TODO
  \end{itemize}
\end{frame}

%% -------------------------------------------------------------------------- %%

\begin{frame}{как работать с фидбеком от геймдизайнеров: админка только тогда, когда без неё станет совсем больно}
  \begin{itemize}
    \item TODO
  \end{itemize}
\end{frame}

%% -------------------------------------------------------------------------- %%

\begin{frame}{что получилось: описание, трудозатраты, оценка успешности}
  \begin{itemize}
    \item TODO
  \end{itemize}
\end{frame}

%% -------------------------------------------------------------------------- %%

\begin{frame}{что дальше: дорога к релизу}
  \begin{itemize}
    \item TODO
  \end{itemize}
\end{frame}

%% -------------------------------------------------------------------------- %%

\begin{frame}{проблемы: геолокация шумит}
  \begin{itemize}
    \item TODO
  \end{itemize}
\end{frame}

%% -------------------------------------------------------------------------- %%

\begin{frame}{проблемы: нет геолокации в зданиях}
  \begin{itemize}
    \item TODO
  \end{itemize}
\end{frame}

%% -------------------------------------------------------------------------- %%

\begin{frame}{проблемы: нет проблем, подстраивайте под них геймплей}
  \begin{itemize}
    \item TODO
  \end{itemize}
\end{frame}

%% -------------------------------------------------------------------------- %%

\begin{frame}{чего не хватает из механизмов: по-крупному --- события, много мелочей}
Например, счётчика пройденных метров.
\begin{itemize}
  \item TODO
\end{itemize}
\end{frame}

%% -------------------------------------------------------------------------- %%

\begin{frame}{ошибки: геолокация появилась поздновато, лишний map}
  \begin{itemize}
    \item TODO
  \end{itemize}
\end{frame}

%% -------------------------------------------------------------------------- %%

\begin{frame}{ошибки: карта в клиенте появилась поздновато}
  \begin{itemize}
    \item TODO
  \end{itemize}
\end{frame}

%% -------------------------------------------------------------------------- %%

\begin{frame}{ошибки: больше выходить на улицу}
  \begin{itemize}
    \item TODO
  \end{itemize}
\end{frame}

%% -------------------------------------------------------------------------- %%

\begin{frame}{ошибки: ещё?}
  \begin{itemize}
    \item TODO
  \end{itemize}
\end{frame}

%% ========================================================================== %%

\section{Вопросы?}

\begin{frame}{Вопросы?}

\begin{center}
\Huge{@agladysh}
\end{center}

\begin{center}
\Large{ag@logiceditor.com}
\end{center}

\begin{center}
\href{http://meetup.com/Lua-in-Moscow}{meetup.com/Lua-in-Moscow}
\end{center}

\end{frame}

%% -------------------------------------------------------------------------- %%

\end{document}
