% Build with xetex.
\documentclass[aspectratio=169,handout,bigger]{beamer}

%% -------------------------------------------------------------------------- %%

\usepackage{xunicode}
\usepackage{xltxtra}
\usepackage{graphicx}

\usepackage{color}

\usepackage{setspace}
\usepackage{ragged2e}

%% -------------------------------------------------------------------------- %%

\usepackage{polyglossia}
\setmainlanguage{russian}
\setotherlanguage{english}

% NB: To get MS fonts for OS X:
%
% 1) Install MS Open XML Converter.
% 2) Install OpenXML_all_fonts.pkg from inside of Converter's mpkg.
%
% http://www.labnol.org/software/tutorials/\
% free-download-calibri-font-on-mac/3684/

\setmainfont{Calibri}
\setsansfont{Calibri}
\setmonofont{Consolas}

%% -------------------------------------------------------------------------- %%

\mode<presentation>{
  \usetheme{default}
}

\useinnertheme{circles}

\setbeamertemplate{navigation symbols}{}
\setbeamertemplate{section in toc}[sections numbered]

\definecolor{chart11}{RGB}{0, 0, 0}
\setbeamercolor{title}{fg=chart11}
\setbeamercolor{author}{fg=chart11}
\setbeamercolor{frametitle}{fg=chart11}
\setbeamercolor{itemize item}{fg=chart11}
\setbeamercolor{itemize subitem}{fg=chart11}
\setbeamercolor{itemize subsubitem}{fg=chart11}
\setbeamercolor{section in toc}{fg=chart11}

\setbeamerfont{title}{series=\bfseries,parent=structure}

\defbeamertemplate*{title page}{customized}[1][]
{
  \vbox{}
  \vfill
  \begingroup
    \centering
    \vspace*{-2.5cm}
    \begin{beamercolorbox}[sep=8pt,left,#1]{title}
      \usebeamerfont{title}\inserttitle\par%
    \end{beamercolorbox}%
    \par
    \begin{beamercolorbox}[sep=8pt,left,#1]{author}
      \usebeamerfont{author}\insertauthor
    \end{beamercolorbox}
    {\usebeamercolor[fg]{titlegraphic}\inserttitlegraphic\par}
  \endgroup
  \vfill
}

%% ========================================================================== %%

\title{Быстрое прототипирование \\
бэкенда игры с геолокацией \\
на OpenResty, Redis и Docker}
\author{Александр Гладыш\\CTO, LogicEditor}
\date{}

%% ========================================================================== %%

\begin{document}

\usebackgroundtemplate{
  \includegraphics[width=\paperwidth,height=\paperheight]{back_title}
}

\maketitle

%% ========================================================================== %%

\usebackgroundtemplate{
  \includegraphics[width=\paperwidth,height=\paperheight]{back}
}

\begin{frame}{Содержание}

\tableofcontents

\end{frame}

%% ========================================================================== %%

\section*{}

\begin{frame}{Обо мне}

\begin{itemize}
\item В разработке ПО с 2002-го,
\item большую часть этого времени --- в геймдеве (разработка, проектирование, управление),
\item также, последние годы --- нагруженные интернет-решения, enterprise ПО и др.
\item Организатор \href{http://meetup.com/Lua-in-Moscow}{meetup.com/Lua-in-Moscow}
\end{itemize}

\end{frame}

%% -------------------------------------------------------------------------- %%

\begin{frame}{О чём доклад: геоигры}
\end{frame}

%% -------------------------------------------------------------------------- %%

\begin{frame}{О чём доклад: конкретный кейс, очень ранний, зачем доклад?}
\end{frame}

%% -------------------------------------------------------------------------- %%

\begin{frame}{О чём доклад: технологии, не геймдизайн и не монетизация}
\end{frame}

%% -------------------------------------------------------------------------- %%

\begin{frame}{О чём доклад: прототипирование vs. продакшен: стадии разработки}
\end{frame}

%% -------------------------------------------------------------------------- %%

\begin{frame}{О чём доклад: прототипирование vs. продакшен: приоритеты разработки}
\end{frame}

%% -------------------------------------------------------------------------- %%

\begin{frame}{О чём доклад: время на вес золота}
\end{frame}

%% -------------------------------------------------------------------------- %%

\begin{frame}{О чём доклад: механизмы, а не решения}
\end{frame}

%% -------------------------------------------------------------------------- %%

\begin{frame}{О чём доклад: минимум кода, плохой код, быстрое итерирование}
\end{frame}

%% -------------------------------------------------------------------------- %%

\begin{frame}{Цель: проверить ряд гипотез по геоиграм, сгенерировать новую, найти где фан}
\end{frame}

%% -------------------------------------------------------------------------- %%

\begin{frame}{Задача: максимально быстро написать бэкенд для быстрого прототипирования, проверяя гипотезы уже во время написания}
\end{frame}

%% -------------------------------------------------------------------------- %%

\begin{frame}{Задача: выявить технические ограничения на проект}
\end{frame}

%% -------------------------------------------------------------------------- %%

\begin{frame}{Задача: базовый клиент тоже будет нужен}
\end{frame}

%% -------------------------------------------------------------------------- %%

\begin{frame}{План работы, обзор}
\end{frame}

%% -------------------------------------------------------------------------- %%

\begin{frame}{выбор геймплея первого прототипа: задачи}
\end{frame}

%% -------------------------------------------------------------------------- %%

\begin{frame}{выбор геймплея первого прототипа: описание геймплея}
\end{frame}

%% -------------------------------------------------------------------------- %%

\begin{frame}{выбор технологии: критерии; это же прототип, Карл!}
\end{frame}

%% -------------------------------------------------------------------------- %%

\begin{frame}{выбор технологии: почему клиент на втором плане?}
\end{frame}

%% -------------------------------------------------------------------------- %%

\begin{frame}{выбор технологии: RESTful сервер на openresty, redis, docker, html5 на клиенте}
\end{frame}

%% -------------------------------------------------------------------------- %%

\begin{frame}{выбор технологии: почему не что-то готовое?}
\end{frame}

%% -------------------------------------------------------------------------- %%

\begin{frame}{технологии: html5 --- тяп-ляп и в прототип, доклад про сервер}
\end{frame}

%% -------------------------------------------------------------------------- %%

\begin{frame}{технологии: redis, geoadd etc.}
\end{frame}

%% -------------------------------------------------------------------------- %%

\begin{frame}{технологии: openresty, nginx + lua}
\end{frame}

%% -------------------------------------------------------------------------- %%

\begin{frame}{технологии: docker, reproducibility, minimum configuration}
\end{frame}

%% -------------------------------------------------------------------------- %%

\begin{frame}{технологии: docker-compose: много докеров}
\end{frame}

%% -------------------------------------------------------------------------- %%

\begin{frame}{docker: how to install docker, docker-compose on Ubuntu}
\end{frame}

%% -------------------------------------------------------------------------- %%

\begin{frame}{docker: архитектура одного прототипа (схема с редисом, nginx)}
\end{frame}

%% -------------------------------------------------------------------------- %%

\begin{frame}{docker: базовые образы redis, openresty/openresty}
\end{frame}

%% -------------------------------------------------------------------------- %%

\begin{frame}{docker: dockerfiles}
\end{frame}

%% -------------------------------------------------------------------------- %%

\begin{frame}{docker: nginx config, трюки}
\end{frame}

%% -------------------------------------------------------------------------- %%

\begin{frame}{docker: nginx config, трюки: lua code cache, routers}
\end{frame}

%% -------------------------------------------------------------------------- %%

\begin{frame}{docker-compose: больше одного прототипа}
\end{frame}

%% -------------------------------------------------------------------------- %%

\begin{frame}{docker-compose: ymls; довольно про сисадминство}
\end{frame}

%% -------------------------------------------------------------------------- %%

\begin{frame}{код: проектирование механизмов для первого прототипа, задачи}
\end{frame}

%% -------------------------------------------------------------------------- %%

\begin{frame}{игровой объект: обзор}
\end{frame}

%% -------------------------------------------------------------------------- %%

\begin{frame}{игровой объект: характеристики}
\end{frame}

%% -------------------------------------------------------------------------- %%

\begin{frame}{игровой объект: прототипы}
\end{frame}

%% -------------------------------------------------------------------------- %%

\begin{frame}{игровой объект: действия}
\end{frame}

%% -------------------------------------------------------------------------- %%

\begin{frame}{игровой объект: права на действия}
\end{frame}

%% -------------------------------------------------------------------------- %%

\begin{frame}{игровой объект: координаты}
\end{frame}

%% -------------------------------------------------------------------------- %%

\begin{frame}{игровой объект: как это ложится на базу, плевать на производительность, плевать на атомарность, плевать на читеров}
\end{frame}

%% -------------------------------------------------------------------------- %%

\begin{frame}{игровой объект: почему так?}
\end{frame}

%% -------------------------------------------------------------------------- %%

\begin{frame}{апи: статус}
\end{frame}

%% -------------------------------------------------------------------------- %%

\begin{frame}{апи: действия}
\end{frame}

%% -------------------------------------------------------------------------- %%

\begin{frame}{апи: отложенные действия}
\end{frame}

%% -------------------------------------------------------------------------- %%

\begin{frame}{апи: нет админки (пока), есть внутриигровые админские действия}
\end{frame}

%% -------------------------------------------------------------------------- %%

\begin{frame}{апи: сброс и патч базы; проще, ещё проще!}
\end{frame}

%% -------------------------------------------------------------------------- %%

\begin{frame}{клиент: первый клиент это curl, демонстрация curl}
\end{frame}

%% -------------------------------------------------------------------------- %%

\begin{frame}{клиент: демонстрация html5}
\end{frame}

%% -------------------------------------------------------------------------- %%

\begin{frame}{клиент: html5 геолокация, https-only (кроме localhost)}
\end{frame}

%% -------------------------------------------------------------------------- %%

\begin{frame}{клиент: как работает версия на js}
\end{frame}

%% -------------------------------------------------------------------------- %%

\begin{frame}{клиент, трюки: имена объектов}
\end{frame}

%% -------------------------------------------------------------------------- %%

\begin{frame}{клиент, трюки: гуглекарта}
\end{frame}

%% -------------------------------------------------------------------------- %%

\begin{frame}{клиент, трюки: асинхронная перерисовка}
\end{frame}

%% -------------------------------------------------------------------------- %%

\begin{frame}{три вектора развития: геймплей, технологии, фичи; приоритеты; не закопаться!}
\end{frame}

%% -------------------------------------------------------------------------- %%

\begin{frame}{как работать с фидбеком от геймдизайнеров: механизмы, не хаки}
\end{frame}

%% -------------------------------------------------------------------------- %%

\begin{frame}{как работать с фидбеком от геймдизайнеров: приоритеты: баги, новые механизмы, далее по убыванию боли}
\end{frame}

%% -------------------------------------------------------------------------- %%

\begin{frame}{как работать с фидбеком от геймдизайнеров: быстрые итерации, садиться рядом и кодить}
\end{frame}

%% -------------------------------------------------------------------------- %%

\begin{frame}{как работать с фидбеком от геймдизайнеров: админка только тогда, когда без неё станет совсем больно}
\end{frame}

%% -------------------------------------------------------------------------- %%

\begin{frame}{что получилось: описание, трудозатраты, оценка успешности}
\end{frame}

%% -------------------------------------------------------------------------- %%

\begin{frame}{что дальше: дорога к релизу}
\end{frame}

%% -------------------------------------------------------------------------- %%

\begin{frame}{проблемы: геолокация шумит}
\end{frame}

%% -------------------------------------------------------------------------- %%

\begin{frame}{проблемы: нет геолокации в зданиях}
\end{frame}

%% -------------------------------------------------------------------------- %%

\begin{frame}{проблемы: нет проблем, подстраивайте под них геймплей}
\end{frame}

%% -------------------------------------------------------------------------- %%

\begin{frame}{чего не хватает из механизмов: по-крупному --- события, много мелочей}
\end{frame}

%% -------------------------------------------------------------------------- %%

\begin{frame}{ошибки: геолокация появилась поздновато, лишний map}
\end{frame}

%% -------------------------------------------------------------------------- %%

\begin{frame}{ошибки: карта в клиенте появилась поздновато}
\end{frame}

%% -------------------------------------------------------------------------- %%

\begin{frame}{ошибки: больше выходить на улицу}
\end{frame}

%% -------------------------------------------------------------------------- %%

\begin{frame}{ошибки: ещё?}
\end{frame}

%% ========================================================================== %%

\section*{Вопросы?}

\begin{frame}

\begin{center}
\Huge{@agladysh}
\end{center}

\begin{center}
\Large{ag@logiceditor.com}
\end{center}

\begin{center}
\href{http://meetup.com/Lua-in-Moscow}{meetup.com/Lua-in-Moscow}
\end{center}

\end{frame}

%% -------------------------------------------------------------------------- %%

\end{document}
