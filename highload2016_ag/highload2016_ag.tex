% Build with xetex.
\documentclass[aspectratio=169,handout,bigger]{beamer}

%% -------------------------------------------------------------------------- %%

\usepackage{xunicode}
\usepackage{xltxtra}
\usepackage{graphicx}

\usepackage{color}

\usepackage{setspace}
\usepackage{ragged2e}

\usepackage[normalem]{ulem}

\newcommand{\soutt}[1]{%
    \renewcommand{\ULthickness}{1pt}%
       \sout{#1}%
    \renewcommand{\ULthickness}{.4pt}% Resetting to ulem default
}

\usepackage{minted}

\usepackage{tikz}
\usetikzlibrary{shapes,arrows,chains,calc,fit,matrix}

%% -------------------------------------------------------------------------- %%

\usepackage{polyglossia}
\setmainlanguage{russian}
\setotherlanguage{english}

% NB: To get MS fonts for OS X:
%
% 1) Install MS Open XML Converter.
% 2) Install OpenXML_all_fonts.pkg from inside of Converter's mpkg.
%
% http://www.labnol.org/software/tutorials/\
% free-download-calibri-font-on-mac/3684/

\setmainfont{Calibri}
\setsansfont{Calibri}
\setmonofont{Consolas}

%% -------------------------------------------------------------------------- %%

\mode<presentation>{
  \usetheme{default}
}

\useinnertheme{circles}

\setbeamertemplate{navigation symbols}{}
\setbeamertemplate{section in toc}[sections numbered]

\definecolor{chart11}{RGB}{0, 0, 0}
\setbeamercolor{title}{fg=chart11}
\setbeamercolor{author}{fg=chart11}
\setbeamercolor{frametitle}{fg=chart11}
\setbeamercolor{itemize item}{fg=chart11}
\setbeamercolor{itemize subitem}{fg=chart11}
\setbeamercolor{itemize subsubitem}{fg=chart11}
\setbeamercolor{section in toc}{fg=chart11}

\setbeamerfont{title}{series=\bfseries,parent=structure}

\defbeamertemplate*{title page}{customized}[1][]
{
  \vbox{}
  \vfill
  \begingroup
    \centering
    \vspace*{-2.5cm}
    \begin{beamercolorbox}[sep=8pt,left,#1]{title}
      \usebeamerfont{title}\inserttitle\par%
    \end{beamercolorbox}%
    \par
    \begin{beamercolorbox}[sep=8pt,left,#1]{author}
      \usebeamerfont{author}\insertauthor
    \end{beamercolorbox}
    {\usebeamercolor[fg]{titlegraphic}\inserttitlegraphic\par}
  \endgroup
  \vfill
}

\defbeamertemplate{footline}{left page number}
{%
  \hspace*{1em}
  \usebeamercolor[fg]{page number in head/foot}%
  \usebeamerfont{page number in head/foot}%
  \insertpagenumber\,/\,\insertpresentationendpage%
  \vskip1em%
}
\setbeamertemplate{footline}[left page number]

%% ========================================================================== %%

\title{Быстрое прототипирование \\
бэкенда игры с геолокацией \\
на OpenResty, Redis и Docker}
\author{Александр Гладыш\\CTO, LogicEditor}
\date{}

%% ========================================================================== %%

\begin{document}

\usebackgroundtemplate{
  \includegraphics[width=\paperwidth,height=\paperheight]{back_title}
}

% Using [plain] to avoid frame number on the title page.
\begin{frame}[plain]
 \titlepage
\end{frame}

%% ========================================================================== %%

\usebackgroundtemplate{
  \includegraphics[width=\paperwidth,height=\paperheight]{back}
}

\begin{frame}{План доклада}

\tableofcontents

\end{frame}

%% ========================================================================== %%

\section*{}

\begin{frame}{Обо мне}

\begin{itemize}
\item В разработке ПО с 2002-го,
\item большую часть этого времени --- в геймдеве (разработка, проектирование, управление),
\item вне геймдева --- нагруженные интернет-решения, enterprise ПО и др.
\item Организатор \href{http://meetup.com/Lua-in-Moscow}{meetup.com/Lua-in-Moscow}
\end{itemize}

\end{frame}

%% ========================================================================== %%

\section{Кейс}

%% ========================================================================== %%

\begin{frame}{Мобильные игры с геолокацией}
  \centering{\includegraphics[height=.75\textheight]{pokemongo}}
\end{frame}

%% -------------------------------------------------------------------------- %%

\begin{frame}{Цели прототипирования в общем}
  \begin{itemize}
    \item Проверить ряд подходов к построению игрового процесса,
    \item выработать новые идеи,
    \item найти, в чём фан, а в чём --- нет.
  \end{itemize}
\end{frame}

%% -------------------------------------------------------------------------- %%

\begin{frame}{Цели технологической части прототипирования}
  \begin{itemize}
    \item С минимальными затратами получить код,
    \item дающий возможность быстро итерироваться по вариантам геймплея.
    \item Прояснить на практике технические ограничения жанра.
  \end{itemize}
\end{frame}

%% -------------------------------------------------------------------------- %%

\begin{frame}{Результаты}
  За два календарных месяца (менее 100 человеко-часов)
  разработан прототип серверной части игры с геолокацией
  и рудиментарный клиент для неё.
  \vspace*{1em}\par
  Ведётся быстрое итерирование по вариантам построения игрового процесса
  и дальнейшая разработка технологической части проекта.
\end{frame}

%% -------------------------------------------------------------------------- %%

\begin{frame}{О чём этот доклад?}
  \begin{itemize}
    \item Доклад --- технологической части проекта,
    \item не о геймдизайне
    \item и не о монетизации.
  \end{itemize}
\end{frame}

%% -------------------------------------------------------------------------- %%

\begin{frame}{Зачем этот доклад?}
  Делать игры с геолокацией сейчас проще чем когда-либо.
  \vspace*{1em}\par
  Я покажу с чего можно начать.
\end{frame}

%% ========================================================================== %%

\section{Задача}

%% ========================================================================== %%

\begin{frame}{Задача}
  \begin{itemize}
    \item Максимально быстро написать сервер для быстрого прототипирования.
    \item Параллельно с сервером реализовать минимальный клиент.
    \item Проверять гипотезы уже во время написания, если возможно.
    \item Выявить основные технические ограничения на проект.
  \end{itemize}
\end{frame}

%% ========================================================================== %%

\section{Философия}

%% ========================================================================== %%

\begin{frame}{Стадии разработки}
  \begin{itemize}
    \item Препродакшен
    \item Продакшен
    \item Поддержка
  \end{itemize}
\end{frame}

%% -------------------------------------------------------------------------- %%

\begin{frame}{Стадии разработки: Препродакшен}
  \begin{itemize}
    \item Препродакшен
    \begin{itemize}
      \item Поиск геймплея, эскизы, выяснение ограничений.
      \item Более глубокая проработка удачных вариантов геймплея.
      \item Подготовка к продакшену выбранного варианта.
    \end{itemize}
  \end{itemize}
\end{frame}

%% -------------------------------------------------------------------------- %%

\begin{frame}{Продакшен: приоритеты разработки}
  \begin{itemize}
    \item\huge{Восприятие проекта}
    \item\LARGE{Устойчивость ко взлому}
    \item\Large{Масштабируемость}
    \item\Large{Производительность}
    \item\large{Стабильность}
    \item\normalsize{Гибкость}
    \item\normalsize{Скорость итерирования}
    \item\normalsize{Простота}
  \end{itemize}
\end{frame}

%% -------------------------------------------------------------------------- %%

\begin{frame}{Препродакшен: время на вес золота}
  \begin{itemize}
    \item\huge{Скорость итерирования}
    \item\large{Гибкость}
    \item\large{Простота}
    \item\normalsize{Восприятие проекта}
    \item\soutt{\small{Стабильность}}
    \item\soutt{\small{Производительность}}
    \item\soutt{\small{Масштабируемость}}
    \item\soutt{\small{Устойчивость ко взлому}}
  \end{itemize}
\end{frame}

%% -------------------------------------------------------------------------- %%

\begin{frame}{Фокус на скорость и лёгкость разработки}
  \begin{itemize}
    \item Механизмы, а не решения.
    \item Лучше быстро чем правильно.
    \item Много коротких итераций.
  \end{itemize}
\end{frame}

%% -------------------------------------------------------------------------- %%

\begin{frame}{Лучше быстро чем правильно. Переориентация перфекционизма}
  \begin{itemize}
    \item Чем меньше кода тем лучше.
    \item Плохой код лучше сложного.
    \item Хаки в механизмах допустимы, хаки в решениях --- нет.
    \item Рефакторить нужно только то, что болит.
  \end{itemize}
\end{frame}

%% ========================================================================== %%

\section{План разработки}

%% ========================================================================== %%

\begin{frame}{Основные этапы разработки}
  \begin{itemize}
    \item Выбор геймплея пилотного прототипа.
    \item Выбор технологического стека.
    \item Реализация минимального пилотного прототипа.
    \item Итерирование с гейм-дизайнерами.
  \end{itemize}
\end{frame}

%% -------------------------------------------------------------------------- %%

\begin{frame}{Выбор геймплея пилотного прототипа: Задачи}
  \begin{itemize}
    \item Максимально простой
    \item но играбельный
    \item повод реализовать все базовые игровые сущности и механизмы прототипа.
  \end{itemize}
\end{frame}

%% -------------------------------------------------------------------------- %%

\begin{frame}{Геймплей первого прототипа}
  \begin{itemize}
    \item Игрок, перемещаясь по карте, ищет расставленных на ней мобов;
    \item найдя моба может попробовать его поймать с заданной вероятностью успеха;
    \item успешная поимка моба увеличивает счётчик в характеристиках игрока;
    \item пойманный моб исчезает с карты;
    \item через некоторое время моб респавнится в том же месте;
    \item администраторы могут добавлять новых мобов на карту.
  \end{itemize}
\end{frame}

%% -------------------------------------------------------------------------- %%

\begin{frame}{Критерии выбора технологического стека}
  \begin{itemize}
    \item Что-то знакомое, на чём можно написать быстро,
    \item или что-то интересное и зажигающее.
    \item Главное не забыть вовремя выбросить весь код.
  \end{itemize}
\end{frame}

%% -------------------------------------------------------------------------- %%

\begin{frame}{Технологический стек}
  \begin{itemize}
    \item Сервер:
    \begin{itemize}
      \item Redis,
      \item OpenResty,
      \item Docker.
    \end{itemize}
    \item Клиент:
    \begin{itemize}
      \item Одностраничное веб-приложение в браузере,
      \item HTML5.
    \end{itemize}
  \end{itemize}
\end{frame}

%% -------------------------------------------------------------------------- %%

\begin{frame}{Почему не что-то готовое?}
  Not Invented Here Syndrome?
  \vspace*{1em}\par
  И да и нет.
\end{frame}

%% -------------------------------------------------------------------------- %%

\begin{frame}{Redis}
  \begin{itemize}
    \item Надёжное, хорошо зарекомендовавшее себя решение.
    \item Работа с координатами из коробки:
    \begin{itemize}
      \item \texttt{GEOADD key longitude latitude member}
      \item \texttt{GEORADIUS key longitude latitude radius m}
    \end{itemize}
    \item Достаточно удобный набор примитивов для хранения игровых объектов.
    \item Хранимые процедуры на Lua.
  \end{itemize}
\end{frame}

%% -------------------------------------------------------------------------- %%

\begin{frame}{OpenResty}
  \begin{itemize}
    \item Дистрибутив nginx с поддержкой Lua, Redis и многим другим из коробки.
    \item Очень быстро работает, достаточно дружелюбен, хорошо поддерживается.
    \item Пригоден как для быстрого прототипирования, так и для продакшена.
  \end{itemize}
\end{frame}

%% -------------------------------------------------------------------------- %%

\begin{frame}{Docker}
  \begin{itemize}
    \item Воспроизводимая кроссплатформенная среда для разработки.
    \item Хорошо снимает боль по настройке окружения разработчика.
    \item Окружение разработчика можно быстро превратить в прототип серверного окружения.
    \item Требует обновления до достаточно свежей версии.
  \end{itemize}
\end{frame}

%% -------------------------------------------------------------------------- %%

\begin{frame}{Браузер, HTML5}
  \begin{itemize}
    \item На начальном этапе сервер важнее.
    \item Бои в augmented reality и прочие рюшечки делать не нужно,
          можно представлять в голове.
    \item На HTML5 можно быстро написать дёшевый и сердитый клиент.
    \item Есть ограниченный (но достаточный) доступ к данным геолокации.
  \end{itemize}
\end{frame}

%% ========================================================================== %%

\section{Docker}

%% ========================================================================== %%

\begin{frame}{Docker: как установить на Ubuntu}
  \begin{itemize}
    \item В Ubuntu, традиционно, старая версия.
    \item За \texttt{wget | sh} больно бьём по рукам.
    \item \texttt{docker} и \texttt{docker-machine}
          устанавливаем из apt-репозитория докера.
    \item \texttt{docker-compose} устанавливаем через \texttt{pip install}.
    \item Про установку на других платформах читаем официальную документацию.
  \end{itemize}
\end{frame}

%% -------------------------------------------------------------------------- %%

\begin{frame}{Docker на машине разработчика}

\begin{center}
\begin{tikzpicture}
  every join/.style={line},    % Default linetype for connecting boxes
  scale=1, transform shape
]

\tikzset{
  basebase/.style={
    align=center, minimum height=2em, minimum width=9em, color=chart11,
    inner sep=.5em, node distance=3em and 11em
  },
  rect/.style={
    basebase, draw, on grid
  },
  coord/.style={coordinate},
  line/.style={->, draw, chart11},
}

\node[rect, rounded corners]                     (client)    {Клиент};
\node[rect, rounded corners, below of=client]    (docker)    {localhost:8080};
\node[rect, below of=docker]    (openresty) {OpenResty};
\node[rect, below of=openresty] (redis)     {Redis};

\draw [line] (client)    -- (docker);
\draw [line] (docker)    -- (openresty);
\draw [line] (openresty) -- (redis);

\end{tikzpicture}
\end{center}

\end{frame}

%% -------------------------------------------------------------------------- %%

\begin{frame}[fragile]{docker-compose.yml для разработки: Redis}
\begin{minted}{yaml}
version: "2"
services:
  redis:
    image: redis
    volumes:
      - ./redis:/data
    command: redis-server --appendonly yes
  openresty: <...>
\end{minted}
\end{frame}

%% -------------------------------------------------------------------------- %%

\begin{frame}[fragile]{OpenResty: интересные места nginx.conf (в сокращении)}
\begin{minted}{nginx}
error_log logs/error.log notice;
http {
  include resolvers.conf;
  lua_package_path "$prefix/lualib/?.lua;;";
  lua_code_cache off; # TODO: Enable on production!
  server {
    listen 8080;
    include mime.types;
    default_type application/json;
    location / { index index.html; root static/; }
    location = /api/v1/ { content_by_lua_file 'api/index.lua'; }
  }
}
\end{minted}
\end{frame}

%% -------------------------------------------------------------------------- %%

\begin{frame}[fragile]{OpenResty: Dockerfile}
\begin{minted}{docker}
FROM openresty/openresty
COPY bin/entrypoint.sh /usr/local/bin/openresty-entrypoint.sh
COPY nginx/conf /usr/local/openresty/nginx/conf
COPY nginx/lualib /usr/loca/openresty/nginx/lualib
COPY nginx/lua /usr/loca/openresty/nginx/lua
COPY nginx/static /usr/loca/openresty/nginx/static
ENTRYPOINT /usr/local/bin/openresty-entrypoint.sh
\end{minted}
\end{frame}

%% -------------------------------------------------------------------------- %%

\begin{frame}[fragile]{OpenResty: entrypoint.sh}
\begin{minted}{bash}
#!/bin/sh
grep nameserver /etc/resolv.conf \
  | awk '{print  "resolver " $2 ";"}' \
  > /usr/local/openresty/nginx/conf/resolvers.conf
/usr/local/openresty/bin/openresty -g 'daemon off;' "$@"
\end{minted}
\end{frame}

%% -------------------------------------------------------------------------- %%

\begin{frame}[fragile]{docker-compose.yml для разработки: OpenResty}
\begin{minted}{yaml}
  <...>
  openresty:
    build: .
    ports:
      - "8080:8080"
    volumes:
      - ./nginx/lualib:/usr/local/openresty/nginx/lualib:ro
      - ./nginx/api:/usr/local/openresty/nginx/api:ro
      - ./nginx/static:/usr/local/openresty/nginx/static:ro
    links:
      - redis
\end{minted}
\end{frame}

%% ========================================================================== %%

\section{HTTP API}

%% ========================================================================== %%

\begin{frame}{Вызовы API}
  \begin{itemize}
    \item Пользовательские вызовы:
      \begin{itemize}
        \item \texttt{/} состояние игрового мира,
        \item \texttt{/go/:go-id/} состояние игрового объекта,
        \item \texttt{/go/:go-id/act/:action-id} выполнение действия.
      \end{itemize}
    \item Системные вызовы:
    \begin{itemize}
      \item \texttt{/register} создание пользователя,
      \item \texttt{/reset} сброс базы в исходное состояние,
      \item \texttt{/patch} апгрейд базы до текущей версии.
    \end{itemize}
    \item NB: Админку (бэкофис) не делаем,
          используем внутриигровые механики для администрирования игрового мира.
  \end{itemize}
\end{frame}

%% ========================================================================== %%

\section{Устройство игрового мира}

%% ========================================================================== %%

\begin{frame}{Игровой объект}
  \begin{itemize}
    \item С точки зрения сервера игровой мир состоит из игровых объектов.
    \item Игровой объект имеет численные характеристики и действия.
    \item Игровые объекты могут иметь координаты.
    \item Игровые объекты без координат должны принадлежать другим объектам
          или быть их прототипами.
  \end{itemize}
\end{frame}

%% -------------------------------------------------------------------------- %%

\begin{frame}{Цепочка прототипов}
  \begin{itemize}
    \item Игровой объект может иметь прототип.
    \item Игровой объект --- прототип в свою очередь также может иметь прототип.
    \item Игровой объект наследует характеристики и действия своих прототипов.
  \end{itemize}
\end{frame}

%% -------------------------------------------------------------------------- %%

\begin{frame}{Характеристики}
  \begin{itemize}
    \item Характеристика --- именованное численное свойство игрового объекта.
    \item Если у игрового объекта нет какой-то характеристики,
          её значение берётся у ближайшего прототипа по цепочке
          (если не нашли --- 0).
  \end{itemize}
\end{frame}

%% -------------------------------------------------------------------------- %%

\begin{frame}{Действия}
  \begin{itemize}
    \item Действие на игровом объекте --- идентификатор из таблицы
          обработчиков действий.
    \item Действие может быть инициировано игроком,
          если у него достаточно на это прав.
  \end{itemize}
\end{frame}

%% -------------------------------------------------------------------------- %%

\begin{frame}[fragile]{Моб: Зелёная Жаба}
\begin{minted}{lua}
{
  id = 'proto.mob.collectable';
  chrs = { 'respawn_dt' = 10 * 60 };
  actions = { 'mob.collect' };
};
{
  id = 'proto.mob.toad.green';
  proto_id = 'proto.mob.collectable';
};
{
  id = 'fa2eb7bca46c11e6be447831c1cebc82';
  proto_id = 'proto.mob.toad.green';
  geo = { lat = 55.7558, lon = 37.6173 };
  chrs = { collect_chance = 0.5 };
};
\end{minted}
\end{frame}

%% -------------------------------------------------------------------------- %%

\begin{frame}[fragile]{Действие: поймать моба}
\begin{minted}{lua}
ACTIONS['mob.collect'] = function(target, initiator)
  if
    math.random() * initiator.chrs.collect_skill >
    target.chrs.collect_chance
  then
    -- Inc number of catches for this mob type
    go_inc_chrs(initiator.id, target.proto_id, 1)
    go_schedule_action_initiation( -- Schedule respawn
      target.chrs.respawn_dt, 'mob.spawn',
      { proto_id = target.proto_id, pos = target.pos },
      initiator.id,
    )
    go_remove(target.id) -- Mob is caught, remove
  end
end
\end{minted}
\end{frame}

%% -------------------------------------------------------------------------- %%

\begin{frame}[fragile]{Выполнение отложенных действий}
\begin{minted}{lua}
local timestamp = os.time()
local action_ids = redis:zrangebyscore('da', '-inf', timestamp)
for i = 1, #action_ids do
  -- Execute action_ids[i] action
end
redis:zremrangebyscore('da', '-inf', timestamp)
\end{minted}
\end{frame}

%% -------------------------------------------------------------------------- %%

\begin{frame}[fragile]{Игрок}
\begin{minted}{lua}
{
  id = 'proto.user';
  chrs = { vision = 100, reach = 50 };
};
{
  id = 'user.1';
  geo = { lat = 55.7558, lon = 37.6173 };
  chrs = { collect_skill = 0.5 };
};
\end{minted}
\end{frame}

%% -------------------------------------------------------------------------- %%

\begin{frame}[fragile]{Предмет: Админская шапка}
\begin{minted}{lua}
{
  id = 'proto.item.wearable';
  actions = {
    ['item.don'] = { enabled = true };
    ['item.doff'] = { enabled = false };
  };
}
{
  id = 'proto.item.admin-hat';
  proto_id = 'proto.item.wearable';
  grants = { 'user.admin' };
  chrs = { 'collect_skill' = 0.25 };
};
\end{minted}
\end{frame}

%% -------------------------------------------------------------------------- %%

\begin{frame}[fragile]{Выдадим админскую шапку пользователю}
\begin{minted}{lua}
local hat = go_new('proto.item.admin-hat')

assert(go_get('user.1').stored[1] == nil)

go_store('user.1', hat.id)

assert(go_get('user.1').stored[1] == hat.id)
\end{minted}
\end{frame}

%% -------------------------------------------------------------------------- %%

\begin{frame}{Хранение ("storage")}
  \begin{itemize}
    \item Игровой объект может "хранить" другие объекты.
    \item Хранимые объекты не "видны" извне хранящего объекта.
    \item Пользователю доступны действия
          непосредственно хранимых им объектов.
    \item Характеристики хранимых объектов никак не влияют
          на характеристики хранящих их объектов.
  \end{itemize}
\end{frame}

%% -------------------------------------------------------------------------- %%

\begin{frame}[fragile]{Действия на админской шапке}
\begin{minted}{lua}
ACTIONS['item.don'] = function(target, initiator)
  go_unstore(initiator.id, target.id)
  go_attach(initiator.id, target.id)
  go_disable_action(target.id, 'item.don')
  go_enable_action(target.id, 'item.doff')
end

ACTIONS['item.doff'] = function(target, initiator)
  go_attach(initiator.id, target.id)
  go_store(initiator.id, target.id)
  go_disable_action(target.id, 'item.doff')
  go_enable_action(target.id, 'item.don')
end
\end{minted}
\end{frame}

%% -------------------------------------------------------------------------- %%

\begin{frame}{Прикрепление / надевание ("attachment")}
  \begin{itemize}
    \item К игровому объекту могут быть "прикреплены" другие объекты.
    \item Прикреплённые объекты видны извне родительского объекта.
    \item Пользователю доступны действия
          прикреплённых непосредственно к нему объектов.
    \item Характеристики прикреплённых объектов прибавляются
          к характеристикам родительских объектов.
  \end{itemize}
\end{frame}

%% -------------------------------------------------------------------------- %%

\begin{frame}[fragile]{Предмет: Спавнилка зелёных жаб}
\begin{minted}{lua}
{
  id = 'proto.item.spawner.toad.green';
  actions = {
    ['mob.spawn'] = {
      requires = { 'user.admin' };
      param = { proto_id = 'proto.mob.toad.green' };
    };
  };
};
\end{minted}
\end{frame}

%% -------------------------------------------------------------------------- %%

\begin{frame}{Права на действия}
  \begin{itemize}
    \item Действие доступно для выполнения только если \texttt{grants} игрока
          содержит все записи из \texttt{requires} действия.
    \item Прикреплённые к игроку ("надетые") предметы
          добавляют ему свои \texttt{grants}.
  \end{itemize}
\end{frame}

%% ========================================================================== %%

\section{Демонстрация}

%% ========================================================================== %%

\begin{frame}{Демонстрация}
  \begin{itemize}
    \item Текущую версию приложения вы можете найти на
          \href{https://geo.logiceditor.com/}{geo.logiceditor.com}.
    \item Ссылка на репозиторий с кодом будет опубликована
          там же на этой неделе, следите за анонсами в Twitter @agladysh.
    \item Самый первый клиент --- всегда \texttt{curl}.
          API можно пощупать им, добавив к адресу приложения
          \texttt{/api/v1}.
  \end{itemize}
\end{frame}

%% ========================================================================== %%

\section{Клиент}

%% ========================================================================== %%

\begin{frame}{Как работает клиент?}
  \begin{itemize}
    \item Инициализируются геолокация и гуглекарты.
    \item Текущая позиция отправляется на сервер.
    \item Сервер возвращает перечень видимых объектов с возможными действиями.
    \item Объекты помечаются маркерами на карте и выводятся под ней вёрсткой.
    \item Ожидаем активации действия пользователем либо смены координат.
  \end{itemize}
  \vspace*{1em}\par
  NB:
  \begin{itemize}
    \item Для генерации имён жаб на основе их идентификаторов
          используется chance.js.
    \item Перерисовку лучше проводить по таймеру,
          вне зависимости от цикла обновления данных.
  \end{itemize}
\end{frame}

%% -------------------------------------------------------------------------- %%

\begin{frame}{Геолокация на HTML5}
  \begin{itemize}
    \item \texttt{navigator.geolocation.watchPosition(callback, options)}.
    \item В Chrome пользуйтесь панелью разработчика Sensors для
          отладки геолокации.
    \item В Chrome по соображением безопасности отключена геолокация
          для протокола HTTP (за исключением сайтов на localhost).
          Используйте HTTPS, например, с сертификатами от Let's Encrypt.
  \end{itemize}
\end{frame}

%% -------------------------------------------------------------------------- %%

\begin{frame}[fragile]{Google Maps}
\begin{minted}{js}
new google.maps.Map(assert(document.getElementById('map')), {
  center: new google.maps.LatLng(pos.lat, pos.lon),
  zoom: 18,
  mapTypeId: google.maps.MapTypeId.ROADMAP,
  disableDefaultUI: true,
  disableDoubleClickZoom: true,
  draggable: false,
  scrollwheel: false,
  styles: [ { featureType: "poi",
    stylers: [ { visibility: "off" } ]
  } ]
});
\end{minted}
\end{frame}

%% ========================================================================== %%

\section{Итоги}

%% ========================================================================== %%

\begin{frame}{Проблемы проекта}
  \begin{itemize}
    \item Шумные данные от GPS.
    \item Крайне низкая точность геолокации в зданиях.
    \item ...
  \end{itemize}
\end{frame}

%% -------------------------------------------------------------------------- %%

\begin{frame}{\soutt{Проблемы} Технические ограничения проекта}
  \begin{itemize}
    \item Шумные данные от GPS.
    \item Крайне низкая точность геолокации в зданиях.
    \item ...
  \end{itemize}

  Нет проблем. Есть ограничения, под которые нужно подстраивать геймплей.
  Часть из них решается технологически. Нужно ли тратить время на это решение
  --- один из вопросов, на которые должен ответить этап препродакшена.
\end{frame}

%% -------------------------------------------------------------------------- %%

\begin{frame}{Недостающие механизмы}
  \begin{itemize}
    \item Самое крупное --- система событий.
    \item Много мелких функций, например счётчик пройденных метров.
  \end{itemize}
\end{frame}

%% -------------------------------------------------------------------------- %%

\begin{frame}{Ошибки}
  \begin{itemize}
    \item Поздновато добавили геолокациию.
    \item Поздновато добавили карту в клиенте.
    \item Мало выходили на улицу чтобы тестировать.
    \item ...
    \item Найдите сами, сравнив код на слайдах с кодом в проекте.
  \end{itemize}
\end{frame}

%% -------------------------------------------------------------------------- %%

\begin{frame}{Итоги}
  \begin{itemize}
    \item Относительно малыми усилиями
    \item мы сделали крошечную мобильную игру с геолокацией
    \item и заложили фундамент для быстрой разработки большого числа
          несложных прототипов
    \item для поиска удачных вариантов геймплея в этом жанре.
  \end{itemize}
\end{frame}

%% -------------------------------------------------------------------------- %%

\begin{frame}{Благодаря чему разработка быстрая?}
  \begin{itemize}
    \item В проекте небольшой объём простого кода,
    \item использующего базовые механизмы и надёжные сторонние решения
    \item для решения большого числа поставленных геймдизайнером задач.
  \end{itemize}

  \begin{itemize}
    \item Аккуратное расширение возможностей этого кода
    \item ещё больше расширит круг задач, которые можно будет решить,
    \item поменяв несколько строк в конфиге.
  \end{itemize}
\end{frame}

%% -------------------------------------------------------------------------- %%

\begin{frame}{Векторы развития проекта}
  \begin{itemize}
    \item \Huge{Новые варианты геймплея}
    \item \large{Новые функции и механизмы}
    \item \normalsize{Решение технических проблем}
  \end{itemize}
\end{frame}

%% -------------------------------------------------------------------------- %%

\begin{frame}{Как работать с гейм-дизайнером на ранних этапах прототипирования?}
  \begin{itemize}
    \item Быстро итерироваться.
          В идеале --- садиться рядом, кодить и сразу получать фидбек.
          Пробовать самому иногда делать рутинную часть работы гейм-дизайнера,
          чтобы лучше понять, что именно мешает и тормозит процесс.
    \item В первую очередь исправлять мешающие тестировать геймплей баги,
          потом улучшать старые механизмы в коде и добавлять новые.
    \item Если для новой геймплейной фичи нет механизма, добавлять его,
          хотя бы в самой грубой форме, а не реализовывать решение в лоб.
    \item Все прочие задуманные изменения в коде, в том числе рефакторинг,
          реализовывать в порядке убывания боли от их отсутствия.
  \end{itemize}
\end{frame}

%% -------------------------------------------------------------------------- %%

\begin{frame}{Дорога к релизу}
  \begin{itemize}
    \item \soutt{Выбросить весь код и написать заново.}
    \item Создать новый проект и вдумчиво вручную
          перенести в него удачные части кода.
    \item Неудачные --- переписать с нуля.
  \end{itemize}
\end{frame}

%% ========================================================================== %%

\section{Вопросы?}

% Using [plain] to hide frame number.
\begin{frame}[plain]{Вопросы?}

\begin{center}
\Huge{@agladysh}
\end{center}

\begin{center}
\Large{ag@logiceditor.com}
\end{center}

\begin{center}
\href{http://meetup.com/Lua-in-Moscow}{meetup.com/Lua-in-Moscow}
\end{center}

\end{frame}

%% -------------------------------------------------------------------------- %%

\end{document}
