\documentclass[handout]{beamer}

\usepackage{minted}
\usepackage[utf8]{inputenc}
\usepackage{graphicx}

\mode<presentation>{
  \usetheme{default}
}

\setbeamertemplate{navigation symbols}{}
\setbeamertemplate{footline}[frame number]

%% -------------------------------------------------------------------------- %%

\title{A visual DSL toolkit in Lua}
\subtitle{Past, present and future}
\author{\includegraphics[height=.4\textheight]{logo}}
\institute{Alexander Gladysh <ag@logiceditor.com>}
\date{Lua Workshop 2013\\Toulouse}

%% -------------------------------------------------------------------------- %%

\begin{document}

\maketitle

%% -------------------------------------------------------------------------- %%

\begin{frame}{Outline}

\frametitle

\tableofcontents

\end{frame}

%% -------------------------------------------------------------------------- %%
\section{Introduction}

\begin{frame}{Alexander Gladysh}

\frametitle

\begin{itemize}
\item CTO, co-founder at LogicEditor
\item In löve with Lua since 2005
\end{itemize}

\end{frame}

%% -------------------------------------------------------------------------- %%

\begin{frame}{LogicEditor}

\begin{itemize}
\item Use Lua to develop:
\begin{itemize}
\item Visual DSL toolkit (subject of this talk)
\item Big-data analytics
\item Intensive-load web-services
\item In-browser and mobile games
\end{itemize}
\item 600+ KLOC of private Lua codebase
\item Some contributions to open-source Lua projects
\end{itemize}

\end{frame}

%% -------------------------------------------------------------------------- %%

\begin{frame}{The Problem}

\begin{itemize}
\item Business-logic is highly volatile
\item Programmers are not domain area specialists
\item Specialist \leftrightarrow Manager \leftrightarrow Programmer loop is slow
\item Specialist \leftrightarrow Programmer loop is expensive
\item Let specialists do the business-logic!
\end{itemize}

\end{frame}

%% -------------------------------------------------------------------------- %%

\begin{frame}

\includegraphics[height=.8\textheight]{omgwtf}

\end{frame}

%% -------------------------------------------------------------------------- %%

\begin{frame}{Non-programmers aren't programmers}

It is not enough to be able to compose an algorithm and even implement it
with some simple programming language.

For commercial programming you'll also need, at least:

\begin{itemize}
\item Technical background
\item Debugging skills
\item Team coding skills
\end{itemize}

\end{frame}

%% -------------------------------------------------------------------------- %%

\begin{frame}{Solution}

\begin{itemize}
\item A tool that prevents \textit{technical} mistakes
\item While limiting creativity as little as possible
\item And is within grasp of a non-programmer.
\end{itemize}

\end{frame}

%% -------------------------------------------------------------------------- %%

\begin{frame}{Ad-hoc implementations}

\begin{itemize}
\item One-shot, very limited flexibility
\item Full of crutches
\item Hard to maintain
\end{itemize}

\end{frame}

%% -------------------------------------------------------------------------- %%

\section{Classic third-party alternatives}

%% -------------------------------------------------------------------------- %%

\begin{frame}{MIT Scratch}

\includegraphics[height=.8\textheight]{scratch}

\end{frame}

%% -------------------------------------------------------------------------- %%

\begin{frame}{Descent Freespace Editor Events}

\includegraphics[height=.8\textheight]{freespace}

\end{frame}

%% -------------------------------------------------------------------------- %%

\begin{frame}{Lego NXT-G}

\includegraphics[height=.8\textheight]{lego}

\end{frame}

%% -------------------------------------------------------------------------- %%

\begin{frame}{Apple Automator}

\includegraphics[height=.8\textheight]{automator}

\end{frame}

%% -------------------------------------------------------------------------- %%

\begin{frame}{What is in common?}

\begin{itemize}
\item \textbf{A visual domain-specific language},
\item that allows user to describe
\item \textbf{the control-flow}.
\end{itemize}

\end{frame}

%% -------------------------------------------------------------------------- %%

\section{Past generations}

%% -------------------------------------------------------------------------- %%

\begin{frame}{A retrospective of ideas}

Screenshots shown here are from editors done by
me and/or my colleagues for different companies
we worked for, over time.

Only an idea was re-used and improved between generations.

\end{frame}

%% -------------------------------------------------------------------------- %%

\begin{frame}{Video-adventure game editor}

(No screenshots available)

\begin{itemize}
\item Legacy, circa 2002—2004
\item Graph of in-game dialog remarks and answer choices
\item Allowing to tie-in video-loop to remark
\item No Lua.
\end{itemize}

\end{frame}

%% -------------------------------------------------------------------------- %%

\begin{frame}{Adventure game dialog editor, I, II}

\includegraphics[height=.4\textheight]{adventure}

\begin{itemize}
\item Graph of in-game dialog remarks and answer choices
\item With ability to add custom Lua code logic (in II)
\item Generated data (in Lua) for a state machine (also in Lua)
\end{itemize}

\end{frame}

%% -------------------------------------------------------------------------- %%

\begin{frame}{Browser MMO quest editor, I, II}

\includegraphics[height=.8\textheight]{browser}

\end{frame}

%% -------------------------------------------------------------------------- %%

\begin{frame}{Browser MMO magic system editor}

\includegraphics[height=.8\textheight]{browser}

\end{frame}

%% -------------------------------------------------------------------------- %%

\begin{frame}{Analysis}

\begin{itemize}
\item Some non-programmers prefer visual control-flow editors.
\item Some — textual representation.
\item (Programmers hate to use both kinds.)
\item All editors were very useful, some — invaluable.
\item But, in retrospective, some should have been replaced by dedicated coders.
\item None of the past-generation editors were flexible enough to be
used outside its immediate domain (but this never was an official goal
for them).
\end{itemize}

\end{frame}

%% -------------------------------------------------------------------------- %%

\section{The present generation}

%% -------------------------------------------------------------------------- %%

\begin{frame}{The Visual Business Logic Editor Toolkit}

\includegraphics[height=.8\textheight]{gamector}

\end{frame}

%% -------------------------------------------------------------------------- %%

\begin{frame}{Design goals}

\begin{itemize}
\item Easy to create new editors.
\item Easy to support existing editors.
\item Easy to integrate with "any" other project on "any" technology.
\item Easy \textit{enough} to learn and use by end-users.
\end{itemize}

\end{frame}

%% -------------------------------------------------------------------------- %%

\begin{frame}{Editor use-cases}

For example:

\begin{itemize}
\item A dialog editor for a game scenario writer.
\item A magic system editor for a game-designer.
\item A mission logic editor for a game level-designer.
\item A DB query editor for a data analyst (Hadoop, anyone?).
\item An advertising campaign targeting editor for a marketer.
\item ...and so on.
\end{itemize}

\end{frame}

%% -------------------------------------------------------------------------- %%

\begin{frame}{Technology}

\begin{itemize}
\item The data is a tree corresponding to the control flow (or to
anything tree-like, actually).
\item The output is structured text (code or data).
\item Editor code, UI and backend, is generated by Lua code in the
Toolkit, from the data "schema".
\item Editor UI is in JavaScript / HTML, backend is in Lua.
\end{itemize}

\end{frame}

%% -------------------------------------------------------------------------- %%

\begin{frame}{The Data Schema}

\begin{itemize}
\item Embedded Lua DSL (see my talk on Lua WS'11).
\item Describes how to:
\begin{itemize}
 \item check data validity,
 \item generate default data,
 \item render data to editor UI,
 \item change data in editor UI,
 \item render the conforming data to the output code (or data).
\end{itemize}
\item Two layers: human-friendly and machine-friendly
\end{itemize}

\end{frame}

%% -------------------------------------------------------------------------- %%

\begin{frame}{Schema Example}[fragile]

\begin{minted}{lua}
lang:root "lua.print-string";

lang:func "lua.print-string" {
  "lua.string.value";
  render:js [[Print string]] {
    [[Print: ${1}]];
  };
  render:lua {
    [[print(${1})]];
  };
}

lang:value "lua.string.value" {
  data_type = "string";
  default = "Hallo, world!";
  render:js [[String Value]] { [[${1}]] };
  render:lua { [[${1}]] };
}
\end{minted}

\end{frame}

%% -------------------------------------------------------------------------- %%

\begin{frame}{Default Data}[fragile]

\begin{minted}{lua}
{
  id = "lua.print-string";
  {
    id = "lua.string.value";
    "Hallo, world!";
  }
}
\end{minted}

Renders to Lua as:

\begin{minted}{lua}
print("Hallo, world!")
\end{minted}

\end{frame}

%% -------------------------------------------------------------------------- %%

\begin{frame}{UI for default data (simplified)}[fragile]

\begin{minted}{lua}
<div id="lua.print-string">
  Print: <span id="lua.string.value">Hallo, world!</span>
</div>
\end{minted}

\textit{NB: That <span> turns to edit-box on click.}

\end{frame}

%% -------------------------------------------------------------------------- %%

\begin{frame}{Extending string type}[fragile]

\begin{minted}{lua}
lang:type "lua.string" {
  init = "lua.string.value";
  render:js [[String]] { menu = [[S]]; [[${1}]] };
  render:lua { [[${1}]] };
}

lang:func "lua.string.reverse" {
  type = "lua.string";
  render:js [[Reverse string]] { [[Reverse: ${1}]] };
  render:lua { [[(${1}):reverse()]] };
}
\end{minted}

\end{frame}

%% -------------------------------------------------------------------------- %%

\begin{frame}{Print with multiple arguments}[fragile]

\begin{minted}{lua}

lang:list "lua.print"
{
  "lua.string";
  render:js [[Print]] {
    empty = [[Print newline]];
    before = [[Print values: <ul><li>]];
    glue = [[</li><li>]];
    after = [[</li></ul>]];
  };
  render:lua {
    before = [[print(]];
    glue = [[,]];
    after = [[)]];
  };
}
\end{minted}

\end{frame}

%% -------------------------------------------------------------------------- %%

\begin{frame}{Main primitives}

\begin{itemize}
\item lang:const
\item lang:value
\item lang:enum
\item lang:func
\item lang:list
\item lang:type
\end{itemize}

\end{frame}

%% -------------------------------------------------------------------------- %%

\begin{frame}{Machine-friendly schema}

\begin{itemize}
node:literal
node:variant
node:record
node:value
node:list
\end{itemize}

\end{frame}

%% -------------------------------------------------------------------------- %%

Data-upgrade routines

\item transformations between two given data versions
\item are done in terms of node schema
\item semi-automatical a stub is generated for many common simpler cases
TODO what else about data upgrade?

TODO what else about the present generation? Consult old talk

\section{The future}

Current generation does its job well, but we see several ways on how to make it better

Several points to improve

+ Better, modern HTML (at the cost of support of IE6)
+ Lua in browser for an optional server-less deployment
+ Even more flexible and expressive Schema DSL

We'll probably go for a control-flow diagram UI first, not text-based one (current text-based is cool enough)

Problems with the current Schema DSL

TODO: Get from the pkle8 memo

Solution

+ Three separate sets of languages:
+ Data format
+ Render to output (per output format)
+ Render to editor (per editor kind)

+ CSS-like rules instead of pre-defined set of node types

Examples

TODO: Get from the pkle8 memo (make sure to use modern ones)

Bonus: New approach to DSLs

DSL as a Finite-state machine

TODO get from the lua-dsl-fsm README, don't forget examples



+ Interested in the current generation of the Toolkit or want to be an early adopter for the next one?

Drop us a line:

toolkit@logiceditor.com


Links, +link to example on GH

%% -------------------------------------------------------------------------- %%

\section{Questions?}

%% -------------------------------------------------------------------------- %%

\begin{frame}{Questions?}

Alexander Gladysh,
ag@logiceditor.com

\end{frame}

%% -------------------------------------------------------------------------- %%

\end{document}

%% -------------------------------------------------------------------------- %%

TODO: Make sure all images are there.
TODO: Validate all code
TODO: Link to a working demo
